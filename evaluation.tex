\chapter{Evaluation}
\label{chap:evaluation}
\section{Testprogramme}
\label{sec:testprogamme}
Um den implementierten Code zu testen und eine reele Einsparung der Verlustleistungsreduktion zu ermitteln wurden die folgenden Assembler-Programme verwendet. Dabei handelt es sich um Programme die eine häufig Anwendung in Hörgerät-Prozessoren finden. Ausschlaggebend fuer die Wahl waren die Anzahl der verwendeten virtuellen Registern da mit einer hohen Zahl an virtuellen Registern das Verbesserungspotential steigt.
\subsection{Beamforming}
Der Beamforming-Algorithmus wird eingesetzt um eine Positionsbestimmung von Schallwellen im Raum durchzuführen und somit eine Ein- bzw Ausblendung von verschiedenen Geräuschen zu gewährleisten und so das Hörerlebnis zu steigern. 

\subsection{emulated floating Point}
Da der verwendete Prozessor keine floating Point Variablen unterstützt, besteht ein Algorithmus der diese emuliert. 

Da die Verlustleistung nicht im Code und zur Laufzeit bestimmt werden kann, musste eine Analyse nach der Ausführung des Codes stattfinden. Hierfür wurde das Tool \"Prime Time Suite\" von Synopsys verwendet. Dieses analysiert alle Schaltaktivitäten im Prozessor und gibt anschließend eine detaillierte Übersicht über die Leistungsaufnahme des Prozessors.\\
Um die Annahme zu bestätigen, dass mit geringerer Schaltaktivitäten an den Adressleitungen der Register Files die Verlustleistung gesenkt werden kann, wurde zuerst die Schaltleistung der Register Adressen untersucht.\\
Des weiteren wurden die Testprogramme zu Beginn manuell gescheduled um die Registerallokation isoliert betrachten zu können. Hierzu kann vor jeder Instruktion angegeben werden in welchem Issue-SLot diese ausgeführt werden soll. Um den Einfluss der Schaltaktivitäten besser bestimmen und nachvollziehen zu können wurden vorerst ausschließlich physikalische Register verwendet. \\
Des weiteren wurden Schreibe- und Lese-Adressen separat voneinander untersucht um den Einfluss der Adressen auf die Schaltleistung weiter aufzuschlüsseln. Dafür wird beispielsweise aus immer der selben Adresse gelesen und in unterschiedliche geschrieben. Das Schaubild XXX Zeigt hierbei die Untersuchung der Schreibadressen.

%\begin{figure}
%	\centering
%	\includesvg[width=1.25\textwidth]{targetswitching}
%	\caption{Schaltleistung Schreibports}
%\end{figure}
Anhand des Schaubildes kann schön erkannt werden, dass mit steigender Hamming-Distanz auch die Schaltleistung steigt. Auffällig ist hierbei der Sprung bei einer Hamming-Distanz von 64. Dieser lässt sich dadurch erklären, dass zwar die Hamming-Distanz identisch ist aber dafuer Adressleitungen verwendet werden die eine größere Lastkapazität aufweisen. Durch die höhere Lastkapazität wird automatisch auch einer größere Leistung benötigt.
Das untenstehende Schaubild XXX zeigt einen Testfall. Hierbei sind die blauen hinterlegen Felder 

\section{Evaluation}
\label{sec:evalutation_verification}