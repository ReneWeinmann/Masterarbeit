
% disable for numbered page 
\thispagestyle{empty}
%
\chapter{Einleitung}
\label{chap:introduction}
In den letzten Jahren steigt die Komplexität von Schaltungen kontinuierlich und scheint dabei dem Gesetz von Gordon Moor zu folgen, welches im Jahre 1965 veröffentlicht wurde. Jedoch hat diese Entwicklung auch eine Kehrseite, denn mit steigender Performance steigt auch gleichzeitig der Energiebedarf. Dies hat zur folge, dass nun der limitierende Faktor bei portablen Geräten bei der Stromaufnahme liegt und nicht mehr bei der Performance. Außerdem sind die Materialien für immer größer werdenden Batterien sehr selten und dementsprechend teuer. Aus diesem Grund ist die Verlustleistungsoptimierung von portablen Geräten ein immer wichtigeres Themengebiet und muss deshalb bestmöglich optimiert werden. Bei mobilen Geräten handelt es sich nicht nur um Smartphones oder Smartwatches sondern auch um medizinische Geräte wie in Fall dieser Arbeit um ein Hörgerät. Durch optimierte Verlustleistung ist es Nutzern möglich längere Zeit besser zu hören, welches die Lebensqualität deutlich erhöht.
In dieser Arbeit soll die Verlustleistung eines solchen Hörgeräteprozessor minimiert werden, umso eine höhere Laufzeit zu ermöglichen. Hierbei soll die explizit die Verlustleistung von Registerspeicherzugriffen optimiert werden. 
Es gibt zwei Methoden die Leistungsaufnahme zu optimieren, zum Einen kann Veränderung der Hardware vorgenommen werden, oder zum Andern kann eine Verbesserung durch Software herbei geführt werden. Eine Optimierung der Hardware ist sehr zeitaufwändig und demzufolge teuer. Außerdem ist es bei vielen Geräten nicht möglich eine weitere Verbesserung durch Anpassen der Hardware sicherzustellen. Aus diesem Grund wird auf eine Optimierung der Software gesetzt. Dies kann insbesondere bei gepipelineten Prozessoren zur Kompilierzeit geschehen. 
Diese Arbeit zeigt wie die Verlustleistung eines solchen Hörgeräteprozessor mithilfe eines genetischen Algorithmus zur Registerallokation optimiert werden kann.

\section{Motivation}
\label{sec:motivation}
BLA BLA

\section{State of the Art}
\label{sec:objectives}
BLA BLA

!Write something
\section{Aufbau der Arbeit}
\label{sec:structure}
For a better reading experience of this thesis, let us provide a short overview of the following chapters.

\hyperref[chap:fundamentals]{\textbf{Chapter 2: Fundamentals}} !Write something\\
\hyperref[chap:architecture]{\textbf{Chapter 3: Architecture}} !Write something\\
\hyperref[chap:evaluation]{\textbf{Chapter 4: Evaluation}} !Write something\\
\hyperref[chap:conclusion]{\textbf{Chapter 5: Conclusion}} !Write something