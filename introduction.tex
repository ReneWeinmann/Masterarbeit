
% disable for numbered page 
\thispagestyle{empty}
%
\chapter{Einleitung}
\label{chap:introduction}
In den letzten Jahren steigt die Komplexität von Schaltungen kontinuierlich und scheint dabei dem Gesetz von Gordon Moore zu folgen, welches im Jahre 1965 veröffentlicht wurde. Jedoch hat diese Entwicklung auch eine Kehrseite, denn mit steigender Komplexität steigt auch gleichzeitig der Energiebedarf. Dies hat zur Folge, dass nun der limitierende Faktor bei portablen Geräten bei der Stromaufnahme liegt und nicht mehr bei der Performance. Außerdem sind die Materialien für die immer größer werdenden Batterien sehr selten und dementsprechend teuer. Aus diesem Grund ist die Verlustleistungsoptimierung von portablen Geräten ein immer wichtigeres Themengebiet und muss deshalb bestmöglich optimiert werden. Bereits bei mobilen Geräten wie Smartphones oder Smartwaches ist dies ein wichtiges Themengebiet, jedoch sind bei medizinischen Geräten, beispielsweise Hörgeräten die Anforderung noch hoeher und ein Trade-Off zwischen Perfomance und Laufzeit muss gefunden werden. Durch optimierte Verlustleistung ist es Nutzern möglich längere Zeit besser zu hören, welches die Lebensqualität deutlich erhöht.
In dieser Arbeit soll die Verlustleistung eines Hörgeräteprozessor minimiert werden, um eine höhere Laufzeit zu ermöglichen. Hierbei soll die explizit die Verlustleistung von Registerspeicherzugriffen untersucht und optimiert werden. 
Es gibt zwei Methoden die Leistungsaufnahme zu optimieren, zum Einen kann Veränderung der Hardware vorgenommen werden, oder zum Andern kann eine Verbesserung durch Software herbei geführt werden. Fuer Hardware wurden bereits Optimierung gefunden. Aus diesem Grund wird auf eine Optimierung der Software gesetzt. Dies kann insbesondere bei Prozessoren mit Pipeline zur Kompilierzeit geschehen. 
Diese Arbeit zeigt wie die Verlustleistung eines solchen Hörgeräteprozessor mithilfe eines genetischen Algorithmus zur Registerallokation optimiert werden kann.

%und  Bei mobilen Geräten handelt es sich nicht nur um Smartphones oder Smartwatches sondern auch um medizinische Geräte wie im Fall dieser Arbeit um ein Hörgerät.

\section{Motivation}
\label{sec:motivation}
Allein in Deutschland tragen 2016 ca. 1,88 Millionen Menschen ein Hörgerät, wobei noch ca. 1,39 Millionen Personen keine Hörhilfe zu tragen, die jedoch rein medizinisch auf dieses angewiesen wären \cite{statistica}. Um diese Zahl zu senken und mehr Hörgeschädigte dazu zu bewegen ein Hörgerät zu tragen, sind die Gerätehersteller damit bemüht die Funktionen und den Tragekomfort weiter zu steigern. Doch auch die Akkulaufzeit spielt bei steigender Komplexität der Funktionen eine immer wichtiger werdende Rolle. Da bis dato die meisten Hörgeräte mit Batterie arbeiten und die Nutzer nicht gewillt sind Funktion gegen Flexibilität einzutauschen, ist es nötig auf diesem Gebiet zu forschen. Ein weiter wichtiger Punkt ist, dass die Gerät-Größen von den Dimensionen eher zu kleineren Apparaturen tendieren. Aus diesem Grund, ist der Einsatz von größeren Batterien ausgeschlossen und es muss eine Lösung gefunden werden, die es erlaubt bei gleichbleibenden Abmessungen den Energieverbrauch zu senken.\\
Mit einem am Institut für Mikroelektronik an der Universität in Hannover entwickeltem Prozessor, sollen dem Hörgeschädigten verbesserte Funktionen zu Verfügung stehen und die Mobilität sowie den Komfort des Trägers erhöhen. Da die Architektur bereits entwickelt und auf Energie verbrauch sowie Chipfläche optimiert wurde, soll nun der Energieverbrauch durch eine Software-Anpassung verbessert werden.
Das Ziel den Batterieverbrauch weitest möglich zu senken um dem Träger einen lästigen Batteriewechsel zu ersparen.

%Energieverbrauch im Register-File am höchsten.
%https://de.statista.com/statistik/daten/studie/252153/umfrage/anzahl-der-hoergeraetetraeger-in-deutschland/ 1.9.2017 10:38uhr
%\section{State of the Art}
%\label{sec:objectives}
%Durch Voruntersuchungen der Verlustleistung war zu erkennen, 

\section{Ziel der Arbeit}
\label{sec:ziele}
Durch Voruntersuchungen wurde gezeigt, dass es einen deutlichen Zusammenhang zwischen der Register-Allokation und der Verlustleistung gibt. Ziel der Arbeit ist es, den Energieverbrauch des Systems durch geeignete Allokation der Register zu minimieren. Hierbei soll untersucht werden welche Faktoren den ausschlaggebenden Faktor für die Leistungsaufnahme bieten. Die Optimierung soll anhand eines genetischen Algorithmus implementiert werden. Hierzu muss eine geeignete Fitness-Funktion ermittelt und die Parameter so angepasst werden, dass eine optimale Lösung in kurzer Rechenzeit ermittelt wird.

%\section{Aufbau der Arbeit}
%\label{sec:structure}
%Um einen besseren Lesefluss zu garantieren, wird nun eine kurze Übersicht über die Arbeit gegeben.
%
%\hyperref[chap:grundlagen]{\textbf{Kapitel 2: Grundlagen}}\\Im Grundlagenteil wird kurz auf den verwendeten Prozessor und die zugrundeliegende Architektur eingegangen. Anschließend wird auf das Scheduling erklärt und die Verlustleistung erläutert. Zum Ende des Kapitels soll ein grober Überblick über die genetischen Algorithmen gegeben werden.\\
%\hyperref[chap:Implementierung]{\textbf{Kapitel 3: Implementierung}}\\Im diesem Kapitel wird das Vorgehen in Arbeit erläutert und die Ergebnisse präsentiert.\\
%\hyperref[chap:evaluation]{\textbf{Kapitel 4: Evaluation}}\\Dieser Abschnitt werden die errungenen Ergebnisse diskutiert und abgewägt.\\
%\hyperref[chap:schlussfolgerung]{\textbf{Kapitel 5: Schlussfolgerung}}\\Zum Ende dieser Arbeit wird eine kurze Schlussfolgerung dargelegt und einen Ausblick in weitere Arbeit gegeben.