
% disable for numbered page 
\thispagestyle{empty}
%
\chapter{Einleitung}
\label{chap:introduction}
Allein in Deutschland trugen 2016 ca. 1,88 Millionen Menschen ein Hörgerät. Ca. 1,39 Millionen Menschen tragen jedoch keine Hörhilfe, obwohl sie aus medizinischer Sicht auf diese angewiesen wären \cite{statistica}. Damit diese Zahl sinkt und mehr Hörgerätgeschädigte zum Hörgerät greifen, sind die Gerätehersteller darum bemüht die Funktionen und den Tragekomfort stetig zu steigern.
Bei der Umsetzung dieser Verbesserungen stehen die Entwickler vor zwei großen Problemstellungen. Zum einen sollen die Geräte eine hohe Flexibiltät in der Anpassung sowie eine Vielzahl an Funktionen aufweisen, zum anderen sollen sie dabei aber auch einen sehr niedrigen Energieverbrauch besitzen \cite{lee2007low}. Die Tendenz geht dabei zu immer komplexeren Höralgorithmen und mehr Anpassungsmöglichkeiten, wobei der limitierende Faktor schlussendlich bei der Stromaufnahme liegt.
Außerdem gilt es zu beachten, dass bis dato die meisten Hörgeräte mit Batterie arbeiten und die Nutzer nicht gewillt sind, Funktion gegen Flexibilität einzutauschen.Deshalb ist es nötig auf diesem Gebiet zu forschen. Ein weiterer wichtiger Punkt ist, dass die Gerät-Größen im Hinblick auf die Dimensionen eher zu kleineren Apparaturen tendieren. Aus diesem Grund ist der Einsatz von größeren Batterien ausgeschlossen und es muss eine Lösung gesucht werden, die es erlaubt, bei gleichbleibenden Abmessungen den Energieverbrauch zu senken. Daraus folgt, dass die Verlustleistungsoptimierung ein immer wichtigeres Themengebiet ist und deshalb bestmöglich optimiert werden muss. 
Zumeist muss dabei ein Abtausch zwischen Performance und Akkulaufzeit in Kauf genommen werden. In dieser Arbeit wird jedoch ein Ansatz präsentiert, welcher die Verlustleistung minimiert und dabei keine Performance einbüßt.\\
Mit einem am Institut für Mikroelektronik der Universität Hannover entwickelten Hörgeräteprozessor sollen den Hörgeschädigten verbesserte Funktionen zu Verfügung stehen und die Mobilität sowie der Komfort des Trägers erhöht werden. Da die Architektur bereits entwickelt und auf Energieverbrauch sowie Chipfläche optimiert wurde, soll nun die Verlustleistung durch eine Software-Anpassung verbessert werden. Anders als bei einer Optimierung durch Hardware muss keine wertvolle Chipfläche eingesetzt werden um die Leistungsaufnahme zu senken. Diese Software-Optimierungen können, insbesondere bei Prozessoren mit Pipeline, zur Compilezeit geschehen.\\
Betrachtet man die Aufteilung des Stromverbrauchs im Prozessor, fällt das Register-File mit ca. 60\% am deutlichsten ins Gewicht. Dies impliziert ein hohes Optimierungspotential und soll demnach näher untersucht werden. In dieser Arbeit wird dieses Potential durch eine geeignete Registeradressierung ausgenutzt, wobei der Fokus auf der Verlustleistungsoptimierung der Registerspeicherzugriffe liegt. Da die Verlustleistung zur Compilezeit nicht gemessen werden kann, muss eine Metrik gefunden werden, die diese annähert. Mittels dieser und mit Hilfe eines genetischen Algorithmus wird die Verlustleistung gesenkt. Das Ziel ist es durch die erwähnten Anpassungen in der Software, den Batterieverbrauch weitest möglich zu senken, um dem Träger einen häufigen Batteriewechsel zu ersparen.


%Durch eine optimierte Verlustleistung ist es Nutzern möglich längere Zeit besser zu hören, welches die Lebensqualität deutlich erhöht. In den letzten Jahren steigt die Komplexität von Prozessoren kontinuierlich und scheint dabei dem Gesetz von Gordon Moore zu folgen, welches im Jahre 1965 veröffentlicht wurde \cite{moore1965moore}. Jedoch hat diese Entwicklung auch eine Kehrseite, denn mit steigender Komplexität, steigt ebenfalls der Energiebedarf der Prozessoren an 
%und  Bei mobilen Geräten handelt es sich nicht nur um Smartphones oder Smartwatches sondern auch um medizinische Geräte wie im Fall dieser Arbeit um ein Hörgerät. Außerdem sind die Materialien für die immer größer werdenden Batterien sehr selten und dementsprechend teuer.Bereits bei mobilen Geräten wie Smartphones oder Smartwaches ist dies ein wichtiges Themengebiet, jedoch sind bei medizinischen Geräten, beispielsweise Hörgeräten die Anforderung noch höher und ein Trade-Off zwischen Perfomance und Laufzeit muss gefunden werden

%\section{Motivation}
%\label{sec:motivation}

%Energieverbrauch im Register-File am höchsten.

%\section{State of the Art}
%\label{sec:objectives}
%Durch Voruntersuchungen der Verlustleistung war zu erkennen, 

\section{Ziel der Arbeit}
\label{sec:ziele}
Durch Voruntersuchungen wurde gezeigt, dass ein deutlicher Zusammenhang zwischen der Register-Allokation und der Verlustleistung in einem Hörgeräteprozessor besteht. Ziel dieser Arbeit ist es, den Energieverbrauch des Systems durch geeignete Allokation der Register zu minimieren. Dafür soll vorerst eine Heuristik implementiert werden, die beweist, dass die Registerzugriffe einen Einfluss auf die Verlustleistung haben. Im Anschluss sollen Untersuchungen zeigen, welche Metrik verwendet werden muss, um die Verlustleistung zur Compilezeit bestmöglich anzunähern. Diese Metrik soll dazu verwendet werden, die Verlustleistung anhand eines genetischen Optimierungsalgorithmus bestmöglich zu minimieren. Dazu müssen die Parameter des Algorithmus so angepasst werden, dass eine optimale Lösung in kurzer Rechenzeit ermittelt wird. Des Weiteren sollen die Ergebnisse mittels einem Verlustleistungsanalyse-Tool für eine 40 nm ASIC-Technologie evaluiert werden. Um dies zu ermöglichen, sind synthetische Tests zu implementieren. Alles in Allem soll gezeigt werden, dass mit einer geeigneten Registeradressierung die Verlustleistung eines DSPs minimiert werden kann und so ohne Performance-Einbuße oder Hardwareanpassung eine Verbesserung herbei geführt werden kann.

%\section{Aufbau der Arbeit}
%\label{sec:structure}
%Um einen besseren Lesefluss zu garantieren, wird nun eine kurze Übersicht über die Arbeit gegeben.
%
%\hyperref[chap:grundlagen]{\textbf{Kapitel 2: Grundlagen}}\\Im Grundlagenteil wird kurz auf den verwendeten Prozessor und die zugrundeliegende Architektur eingegangen. Anschließend wird auf das Scheduling erklärt und die Verlustleistung erläutert. Zum Ende des Kapitels soll ein grober Überblick über die genetischen Algorithmen gegeben werden.\\
%\hyperref[chap:Implementierung]{\textbf{Kapitel 3: Implementierung}}\\Im diesem Kapitel wird das Vorgehen in Arbeit erläutert und die Ergebnisse präsentiert.\\
%\hyperref[chap:evaluation]{\textbf{Kapitel 4: Evaluation}}\\Dieser Abschnitt werden die errungenen Ergebnisse diskutiert und abgewägt.\\
%\hyperref[chap:schlussfolgerung]{\textbf{Kapitel 5: Schlussfolgerung}}\\Zum Ende dieser Arbeit wird eine kurze Schlussfolgerung dargelegt und einen Ausblick in weitere Arbeit gegeben.