
% disable for numbered page 
\thispagestyle{empty}
%
\chapter{Einleitung}
\label{chap:introduction}
Allein in Deutschland tragen 2016 ca. 1,88 Millionen Menschen ein Hörgerät, wobei noch ca. 1,39 Millionen Personen keine Hörhilfe tragen, die jedoch rein medizinisch auf dieses angewiesen wären \cite{statistica}. Um diese Zahl zu senken und mehr Hörgeschädigte dazu zu bewegen ein Hörgerät zu tragen, sind die Gerätehersteller damit bemüht die Funktionen und den Tragekomfort weiter zu steigern.
Dabei stehen die Entwickler vor zwei grossen Problemstellungen, zum Eine sollen die Geraete eine hohe Flexibilitaet und eine Vielzahl an Funktionen aufweisen, zum Anderen einen sehr niedrigen Energieverbauch haben.
Da bis dato die meisten Hörgeräte mit Batterie arbeiten und die Nutzer nicht gewillt sind Funktion gegen Flexibilität einzutauschen, ist es nötig auf diesem Gebiet zu forschen. Ein weiter wichtiger Punkt ist, dass die Gerät-Größen von den Dimensionen eher zu kleineren Apparaturen tendieren. Aus diesem Grund, ist der Einsatz von größeren Batterien ausgeschlossen und es muss eine Lösung gefunden werden, die es erlaubt bei gleichbleibenden Abmessungen den Energieverbrauch zu senken. Aus diesem Grund ist die Verlustleistungsoptimierung ein immer wichtigeres Themengebiet und muss deshalb bestmoeglichst optimiert werden. 
Zumeist muss dabei ein Abtausch zwischen Perfomance und Akkulaufzeit gefunden werden, in dieser Arbeit wird jedoch ein Ansatz praesentiert, welcher die Verlusleistung minimiert und dabei keine Perfomance einbuest.\\
Mit einem am Institut für Mikroelektronik an der Universität in Hannover entwickeltem Hoergeraeteprozessor, sollen dem Hörgeschädigten verbesserte Funktionen zu Verfügung stehen und die Mobilität sowie den Komfort des Trägers erhöhen. Da die Architektur bereits entwickelt und auf Energie verbrauch sowie Chipfläche optimiert wurde, soll nun der Energieverbrauch durch eine Software-Anpassung verbessert werden. Anders als bei einer Optimierung durch Hardware, muss keine wertvolle Chipfläche eingesetzt um die Leistungsaufnahme zu senken. Diese Software-Optimierungen kann insbesondere bei Prozessoren mit Pipeline zur Kompilierzeit geschehen.\\
Betrachtet man die Aufteilung des Stromverbrauchs im Prozessor fällt das Register-File mit ca. 60\% am deutlichsten ins Gewicht, dies impliziert ein hohes Optimierungspotential und soll demnach untersucht werden. In dieser Arbeit wird dieses Potential durch einen geeignete Registeradressierung ausgenutzt. Dabei liegt der Fokus auf der Verlustleistungsoptimierung der Registerspeicherzugriffe. Da die Verlustleistung zur Compilezeit nicht gemessen werden kann, muss eine Metric gefunden werden die diese annaehert. Mittels dieser wird mit Hilfe eines genetischen Algorithmus die Verlustleistung gesenkt. Das Ziel ist es den Batterieverbrauch weitest möglich zu senken, um dem Träger einen lästigen Batteriewechsel zu ersparen.


%Durch eine optimierte Verlustleistung ist es Nutzern möglich längere Zeit besser zu hören, welches die Lebensqualität deutlich erhöht. In den letzten Jahren steigt die Komplexität von Prozessoren kontinuierlich und scheint dabei dem Gesetz von Gordon Moore zu folgen, welches im Jahre 1965 veröffentlicht wurde \cite{moore1965moore}. Jedoch hat diese Entwicklung auch eine Kehrseite, denn mit steigender Komplexität, steigt ebenfalls der Energiebedarf der Prozessoren an 
%und  Bei mobilen Geräten handelt es sich nicht nur um Smartphones oder Smartwatches sondern auch um medizinische Geräte wie im Fall dieser Arbeit um ein Hörgerät. Außerdem sind die Materialien für die immer größer werdenden Batterien sehr selten und dementsprechend teuer.Bereits bei mobilen Geräten wie Smartphones oder Smartwaches ist dies ein wichtiges Themengebiet, jedoch sind bei medizinischen Geräten, beispielsweise Hörgeräten die Anforderung noch höher und ein Trade-Off zwischen Perfomance und Laufzeit muss gefunden werden

%\section{Motivation}
%\label{sec:motivation}

%Energieverbrauch im Register-File am höchsten.

%\section{State of the Art}
%\label{sec:objectives}
%Durch Voruntersuchungen der Verlustleistung war zu erkennen, 

\section{Ziel der Arbeit}
\label{sec:ziele}
Durch Voruntersuchungen wurde gezeigt, dass es einen deutlichen Zusammenhang zwischen der Register-Allokation und der Verlustleistung gibt. Ziel der Arbeit ist es, den Energieverbrauch des Systems durch geeignete Allokation der Register zu minimieren. Hierbei soll untersucht werden welche Faktoren den ausschlaggebenden Faktor für die Leistungsaufnahme bieten. Die Optimierung soll anhand eines genetischen Algorithmus implementiert werden. Dazu muss eine geeignete Fitness-Funktion ermittelt und die Parameter so angepasst werden, dass eine optimale Lösung in kurzer Rechenzeit ermittelt wird. Des weiteren sollen die Ergebnisse mittels einem Verlustleistungsanalyse-Tool 40nm ASIC evaluiert werden.

%\section{Aufbau der Arbeit}
%\label{sec:structure}
%Um einen besseren Lesefluss zu garantieren, wird nun eine kurze Übersicht über die Arbeit gegeben.
%
%\hyperref[chap:grundlagen]{\textbf{Kapitel 2: Grundlagen}}\\Im Grundlagenteil wird kurz auf den verwendeten Prozessor und die zugrundeliegende Architektur eingegangen. Anschließend wird auf das Scheduling erklärt und die Verlustleistung erläutert. Zum Ende des Kapitels soll ein grober Überblick über die genetischen Algorithmen gegeben werden.\\
%\hyperref[chap:Implementierung]{\textbf{Kapitel 3: Implementierung}}\\Im diesem Kapitel wird das Vorgehen in Arbeit erläutert und die Ergebnisse präsentiert.\\
%\hyperref[chap:evaluation]{\textbf{Kapitel 4: Evaluation}}\\Dieser Abschnitt werden die errungenen Ergebnisse diskutiert und abgewägt.\\
%\hyperref[chap:schlussfolgerung]{\textbf{Kapitel 5: Schlussfolgerung}}\\Zum Ende dieser Arbeit wird eine kurze Schlussfolgerung dargelegt und einen Ausblick in weitere Arbeit gegeben.