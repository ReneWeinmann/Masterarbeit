\chapter{Implementierung}
\label{chap:Implementierung}

Nach genauerem betrachten der Formel für die dynamische Verlustleistung \ref{eq:dynVerlustleistung} hängt diese von vier Faktoren ab, Schaltaktivitäten,  Lastkapazität, Versorgungsspannung und der Frequenz. Die statische Verlustleistung wird nicht weiter betrachtet, da diese nur durch eine Veränderung der Hardware verbessert werden könnte. Dies ist jedoch nicht Teil dieser Arbeit. Da ebenfalls die Spannung und die Frequenz von der Architektur und anderen nicht beeinflussbaren Faktoren vorgegeben wurde kann auch an diesen Parametern nichts geändert werden. Im folgenden wird nun erst darauf eingegangen wie, durch Minimierung der Schaltaktivitäten, die Verlustleistung optimiert werden kann. Im Weiteren Verlauf des Textes wird dann ebenfalls der Einfluss der Lastkapazitäten überprüft.\\
Um in ein Register zu schreiben, muss eine geeignete Adresse an den Adressbus angelegt werden. Da es sich bei der Architektur um 32-bit-Register handelt muss hierfür eine 5-bit Adresse angelegt werden. Dieser Adressbus wird nun auf Schaltaktivität optimiert. Schreibt beispielsweise eine Anweisung an die Adresse Null und die nachfolgende Instruktion an die Adresse 31, so wäre dies der Worst-Case, da in diesem Fall alle 5-Bit umgeladen werden müssten. Befinden sich im Code jedoch virtuelle Register so können diese an beliebiger Stelle zugewiesen und somit die Adresse frei wählbar machen. Dadurch kann die Schaltaktivitäten der Leitungen verringert werden. Aus diesem Grund wurde zu Beginn eine Heuristik entwickelt bei der die Hammingdistanz der Adressleitungen minimiert wird.
\section{Heuristik}
\label{sec:Heuristik}
Um die Hammingdistanz zu berechnen werden die Adressen der letzten Anweisung gespeichert. Da die Architektur zwei Issue-Slots aufweist und diese auf beide Register-Files zugreifen können muss es dementsprechend vier Target-Adressleitungen geben(siehe Abbildung XXX). Wollen beide Instruktionen an das selbe Register-File schreiben, so werden entweder die alle zehn Adressleitungen für die Adressierung verwendet. Verwenden beiden Instruktionen unterschiedliche Register werden jeweils die unteren fünf Adressleitungen des jeweiligen Registers verwendet. Die Adressen bleiben solange in an der Adressleitung angelegt, bis ein neuer Befehl diese Leitungen benutzen muss. Die Distanz wird nun immer mit den zuletzt verwendeten Adressen bestimmt. 