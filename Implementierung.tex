\chapter{Implementierung}
\label{chap:Implementierung}

Nach genauerem betrachten der Formel fuer die dynamische Verlustleistung haengt diese von vier Faktoren ab, Spannung, Schaltaktivitaeten, Frequenz und der Lastkapazitaet. Die statische Verlustleistung wird nicht weiter betrachtet, da diese nur durch eine Veraenderung der Hardware verbessert werden koennte und dies ist nicht Teil dieder Arbeit. Da ebenfalls die Spannung und die Frequenz von der Architektur und anderen Funktionen vorgegeben wurde kann auch an diesen Parametern nichts geaendert werden. Im folgenden wird nun erst darauf eingegangen wie die Verlustleistung optimiert werden kann durch minimierung der Schaltaktiviaeten. Im weitern verlauf des Textes wird dann ebenfalls der Einfluss der Lastkapazitaeten ueberprueft.
Um in ein Register zu schreiben, muss eine geeignete Adresse an den Adressbus angelegt werden. Da es sich bei der Architektur um 32bit-Register handelt muss hierfuer eine 5-bit Adresse angelegt werden. Dieser Adressbus kann nun auf Schaltaktivitaet optimiert werden. Schreibt beispielsweise eine Instruktion an die Adresse Null und die nachfolgende Instruktion an die Adresse 31, so waere dies der Worst-Case, da in diesem Fall alle 5-Bit umgeladen werden muessten. Befinden sich im Code jedoch virtuelle Register so koennen diese an beliebiger Stelle allokiert und somit die Schaltaktivitaeten der Leitungen verringert werden. Aus diesem Grund wurde zu Beginn eine Heuristik entwickelt bei der die Hammingdistanz der Adressleitungen verringert wird.
\section{Heuristik}
\label{sec:Heuristik}
Um die Hammingdistanz der Adressen zu berechnen wird zu beginn immer die letzten Adressen der Instruktionen gespeichert. Da die Architektur zwei Issue Slots aufweist und diese auf beide Regiser-Files zugreifen koennen muess es dementsprechend vier Targetadressleitungen geben.