\chapter{Implementierung}
\label{chap:Implementierung}
Nachdem die nötigen Grundlagen erläutert wurden soll in diesem Kapitel aufgezeigt werden wie der Compiler aufgebaut ist und wie das gesetzte Ziel implementiert wurde. Hierzu wird kurz auf den Ausgangsstatus des Codes und dessen Bestandteile eingegangen um das komplette System besser verstehen zu können. Anschließend wird veranschaulicht wie die Verlustleistung gesenkt werden soll und wie dies mit einer Heuristik implementiert wurde. Im Anschluss wird diese Implementierung durch den Einsatz eines genetischen Algorithmus noch verbessert und die einzelnen Komponenten erläutert.
\section{Ausgangsstatus}
 Das unten stehende Schaubild (siehe Abbildung \ref{fig:flow_compiler}) zeigt den Aufbau des Codes um die Register Allokation besser in den Code einzuordnen. Gestartet wird mit einem Assembler-Programm das in einer ASM-Datei vorliegt und dem Code übergeben wird.


\begin{scriptsize}
	\begin{figure}[htbp] 
		\centering
		\includesvg[width=0.70\textwidth]{flowchart}
		\caption{Compiler Ablaufdiagramm}
		\label{fig:flow_compiler}
	\end{figure}
\end{scriptsize}

\begin{itemize}
	\item Precompiler\\
		Da der Assembler-Code auch Kommentare oder Bestandteile aufweisen kann die nicht in der eigentlichen Assemblersprache enthalten sind, ist es nötig diese aufzulösen. Hierzu dient der Precompiler oder Preprozessor der diese Konstrukte in einen vom Prozessor ausführbaren Code bzw. Anweisungen umwandelt. Ein Beispiel für ein solches Konstrukt ist das An- und Ausschalten des Schedulings für bestimmte Bereiche des Codes. Hierbei kann dem Compiler, im Assembler-Programm durch Angabe des 'scheduling-off'-Befehls, mitgeteilt werden welche SLMs vom Scheduling unberührt bleiben soll.
	\item Virtual Renaming\\
		Im Teil des virtuellen Renaming(Umbenennen) werden die im Assembler-Code vorhandenen virtuellen Register umbenannt, um eine konsistente Allokation und bessere Nachvollziehbarkeit gewährleisteten. Außerdem wird in diesem Teil des Compilers überprüft ob ein virtuelles Register verwendet wurde bevor es zugewiesen ist.
	\item Register Initalisierung\\
		Bei der Register Initalisierung wird über alle Operationen iteriert. Dabei werden die Adressen des Register-Files blockiert, welche bereits von physikalischen Registern verwendet werden. Dadurch wird verhindert, dass diese Positionen im weiteren Verlauf für das virtuelle Register-Mapping verwendet werden können.
	\item Scheduling\\
		Wie in den Grundlagen \ref{sec:scheduling} beschrieben  ist das Scheduling zuständig für die Anordnung der Operationen im Prozessor. Dabei wird je nach gewähltem Optimierungsgrad ein Scheduling gewählt. Hierbei kann zwischen einem Genetischen Instruktions-Scheduling und einem Heuristischem-Ansatz entscheiden werden. Beide Algorithmen rufen intern die Register-Allokation auf. Wird der genetische Algorithmus verwendet wird der Fitness-Wert der Register-Zuweisung mit der Bewertungsfunktion des Instruktion-Schedulings verrechnet.
	\item Register Allokation\\
		Die Register-Allokation ist dafür zuständig,dass alle Register, sowohl physikalisch als auch virtuelle im Register-File abgespeichert werden können. Dazu wird diese nach jedem Scheduling einer SLM aufgerufen und durchläuft hierbei alle Instruktionen sowie die darin verwendeten Register. Bei Verwendung eines physikalischen Registers überprüft die Allokation, ob die vom Programmierer gewählte Register-Adresse und Register-File bereits blockiert oder nicht zuweisbar ist. Handelt es sich um ein virtuelles Register wird zuerst eine geeignete Adresse zugewiesen. Hierbei wird ebenfalls darauf geachtet das diese nicht blockiert oder nicht möglich ist.
		Bei der Regsiter-Allokation bestehen ebenfalls zwei Arten der Zuweisung. Zum Einen eine Heuristik (siehe Kapitel \ref{sec:Heuristik}) zum Andern ein Genetischer Algorithmus (siehe Kapitel \ref{sec:genetischerAlgorithmus}). Beide Alogithmen werden im folgenden nun näher beleuchtet.
\end{itemize}


\section{Ansatz}
\label{chap:ansatz}
Nach genauerem betrachten der Formel für die dynamische Verlustleistung \ref{eq:dynVerlustleistung} hängt diese von vier Faktoren ab, Schaltaktivitäten,  Lastkapazität, Versorgungsspannung und der Frequenz. Die statische Verlustleistung wird nicht weiter betrachtet, da diese nur durch eine Veränderung der Hardware verbessert werden könnte. Dies ist jedoch nicht Teil dieser Arbeit. Da ebenfalls die Spannung und Frequenz von der Architektur und anderen nicht beeinflussbaren Faktoren vorgegeben wurde kann auch an diesen Parametern nichts verbessert werden. Im folgenden wird nun erst darauf eingegangen wie, durch Minimierung der Schaltaktivitäten, die Verlustleistung optimiert werden kann. Im Weiteren Verlauf des Textes wird dann ebenfalls der Einfluss der Lastkapazitäten überprüft.\\
Um in ein Register zu schreiben, muss eine geeignete Adresse an den Adressbus angelegt werden. Da es sich bei der Architektur um 32-bit-Register handelt muss hierfür eine 5-bit Adresse angelegt werden.
Da der Prozessor über zwei Issue-Slots verfügt, müssen pro Slot ein Target- und zwei Source-Register adressiert werden. Hierbei können die Instruktionen auf Register-File 0 oder 1 schreiben bzw. lesen. Aus diesem Grund besitzt jedes der Register-Files vier Lese- und zwei Schreibports. Die Zuordnung der Issues auf die Register können sie aus den Tabellen TTT entnehmen. Schreibt nun beispielsweise eine Instruktion an die Adresse Null und die nachfolgende Instruktion an die Adresse 31 des selben Ports, so wäre dies der Worst-Case, da in diesem Fall alle 5-Bit umgeladen werden müssten.\\
Durch den Einsatz von virtuelle Register \ref{sub:virtuelleR} können Instruktionen an beliebiger Adressen und Register-Files schreiben bzw. lesen. Durch diese Änderung ist es möglich die Schaltaktivitäten des Adressbus zu verringern.\\
In der bestehenden Implementierung wurde diese Zuweisung der Register so geregelt, dass die Register-Files immer von einer Seite befüllt wurden. Die Auswahl des Register-Files wurde hierbei über die Anzahl der freien Register-Adressen ermittelt, und dasjenige File mit der höheren Anzahl an freien Adressen verwendet. Außerdem wurde bei der Zuweisung darauf geachtet, dass es weiterhin möglich ist X2-Befehle und MAC-Operationen zu bedienen.\\
Um nun dieses Mapping auf die Schaltaktivität und somit auch auf die Verlustleistung zu optimieren wurde eine neue Heuristik implementiert.
Diese optimiert die Anzahl der Schaltzyklen anhand der Hamming-Distanz.
Die Hamming-Distanz ist nach dem amerikanischen Mathematiker Richard Wesley Hamming benannt und gibt ein Maß für die Unterschiedlichkeit zweier Zeichenketten an. Hierbei ist die Hamming-Distanz die Anzahl der unterschiedlichen Stellen beider Codewörter.
\begin{equation}
00100 \text{ und } 00001 -> \text{Hamming-Distanze}= 2
\label{eq:hammingdistanze}
\end{equation}
 
\section{Heuristik}
\label{sec:Heuristik}
Um die Hamming-Distanz berechnen zu können, werden die Adressen der letzten Instruktionen gespeichert. Da die Architektur wie bereits erwähnt zwei Issue-Slots aufweist und diese auf beide Register-Files zugreifen können, müssen dementsprechend vier Target- und acht Source-Adressen abgespeichert werden(siehe Abbildung \ref{fig:reg_orga}).\\
Hierbei können lediglich die Adressen innerhalb einer SLM optimiert werden, da das Scheduling über die Anordnung der SLM entscheidet und die Register-Allokation diesem untergeordnet ist. Außerdem ist eine Optimierung bei mehrfacher Ausführung von SLMs, wie es beispielsweise bei Schleifen zustande kommt, nicht möglich. Da SLMs meist sehr groß ausfallen, fällt die nicht optimalen Adress-Toggel nicht stark ins Gewicht. 
Durch den Einsatz einer Architektur mit Adresse-Isolation \ref{subsec:add_iso}, bleiben die zuvor verwendeten Adressen an den Register-File Ports angelegt. Um die Schaltaktivität zu minimieren, wird nun bei jedem virtuellen und physikalischem Register die Hamming-Distanz zu der zuletzt angelegten Adresse des jeweiligen Ports berechnet. Hierbei wird die Adresse auf das Register gemappt welches die geringste Hamming-Distanz zur zuletzt verwendeten Adresse des selbingen Ports aufweist.\\

\begin{algorithm}[H]
	\begin{algorithmic}[1]
		\STATE {:0 \textbf {ADD} V0R0 V0R0 V0R0 \hspace{50pt}:1 \textbf {OR} V1R0 V0R0 V0R0}
		\STATE {:0 \textbf {ADD} V0R2 V0R0 V0R0 \hspace{50pt}:1 \textbf {OR} V1R2 V0R0 V0R0}
		\STATE {:0 \textbf {ADD} V0R4 V0R0 V0R0 \hspace{50pt}:1 \textbf {OR} V1R4 V0R0 V0R0}
		\STATE {:0 \textbf {ADD} VxR0 V0R0 V0R0 \hspace{50pt}:1 \textbf {OR} VxR1 V0R0 V0R0}
		\STATE {:0 \textbf {ADD} VxR2 VxR0 VxR1 \hspace{50pt}:1 \textbf {OR} VxR0 VxR0 VxR1}
		\caption{Heuristik Beispiel}
		\label{code:heuristik}
\end{algorithmic}
\end{algorithm}

\begin{figure}[htbp] 
	\centering
	\includesvg[width=\textwidth]{heuristik}
	\caption{Heuristik Beispiel}
	\label{fig:heuristik}
\end{figure}

Anhand des Codebeispiels \ref{code:heuristik} und der Abbildung \ref{fig:heuristik} mit den bereits allokierten Register soll nun das Mapping der virtuellen Register verdeutlicht werden.
Der Code zeigt hierbei einen bereist geschedueleten Assembler-Code, wobei das Scheduling in diesem Fall, durch die Angabe des Issue-Slots vor der eigentlichen Instruktion, vorgegeben wurde. Außerdem werden in diesem Beispiel für ein besseres Verständis nur Target-Register betrachtet.
Vorerst werden in beiden Register-Files drei Adressen mittels physikalischer Register belegt/blockiert(Abbildung \ref{fig:heuristik} blau hinterlegte Felder). Im Anschluss beginnt die Zuweisung der virtuellen Register(Zeile 4).
Dabei wird mit dem ersten Issue-Slot, der ADD-Operation begonnen. Hierbei  überprüft Algorithmus nun in welchem der Register-Files mehr freie Register zu Verfügung stehen, um eine gleiche Auslastung zu gewährleisten. Für die Zuweisung wird das Register-File mit der höheren Anzahl an freien Registern verwendet. Sind die Zahlen identisch, so wird das Register ausgewählt welches zuletzt zugewiesen wurde. In diesem Fall haben beide Register-Blöcke noch 29 freie Register, deshalb ist das ausgewählte Register-File 0, da zuletzt das Source-Register V0R0 an den Adress-Port geschrieben wurde.
Aus diesem Grund beginnt die Suche eines geeigneten Registers in genau diesem Register-File. Gestartet wird in der neuen Heuristik mit der ersten Adresse, dabei wird überprüft ob diese von einem anderen Register bereits blockiert wurde. Da in der ersten Zeile des Codes diese Adresse V0R0 beschrieben wurde, kann diese nicht als Mapping für das virtuelle Register VxR0 dienen. Deshalb sucht der Algorithmus weiter nach einer geeigneten Zuweisung. Mit der zweiten Adresse ist dies möglich. Anhand dieser wird nun die Hamming-Distanz zur letzten zugewiesenen Target-Adresse auf dem selben Port berechnet. Dabei wird anhand der Tabelle TTT die letzte Adresse ermittelt. Für dieses Beispiel ist dies trivial denn Issue-Slot 0 schreibt immer an Register-File 0 und Issue-Slot 1 an das benachbarte Register-File. Deshalb wird die Hamming-Distanz aus den beiden Adressen vier und eins ermittelt(\ref{eq:hammingdistanze}).Das Ergebnis sowie die Adresse werden gespeichert, falls es bis zu diesem Zeitpunkt noch keine bessere Lösung gab. Anschließend wird die Suche fortgeführt. Dieser Vorgang wird für alle Adressen des Register-Files durchgeführt, es sei denn es wird eine Lösung gefunden bei der die Hamming-Distanz null beträgt. Dieser Fall tritt auf, wenn virtuelle Register zu einer Adresse zugewiesen wurden und die Lebensdauer dieser Variable erlischt. Dadurch wird diese Adresse freigegeben und kann erneut zugewiesen werden.  Werden mehrere Lösungen für eine Allokation gefunden verwendet der Algorithmus die zuerst gefundene Adresse. Im Codebeispiel \ref{code:heuristik} weist der Code die Adresse 5 zu (Abbildung \ref{fig:heuristik} rot hinterlege Felder), da diese eine Hamming-Distanz von eins aufweist.
Die nachfolgende Instruktion wird auf Grund des Gleichgewichts an die selbe Adresse des Register-Files 1 zugewiesen. Dieser Vorgang wird für alle virtuellen Register vorgenommen. Ist bei der Suche keine geeignete Stelle im zuerst gewählten Register-File zu finden, wird das jeweilig Andere Register-File durchsucht. Ist weiterhin keine Zuweisung möglich, schlägt der Algorithmus fehl. Dieser Fall kann jedoch nur auftreten, wenn zu viele Variablen im Assembler-Code verwendet wurden und diese nicht freigegeben werden können. Lösbar ist dieses Problem nur durch eine Anpassung des Assembler-Programms.
In der alten Heuristik-Version wurden die Registerblöcke von einer Seite aus aufgefüllt. Hierbei wurde ebenfalls überprüft ob es sich um eine freie Adresse handelt. Dabei wurde jedoch vernachlässigt wie viele Adressleitungen toggeln. Dementsprechend wurden die Adresse 31 und 30 des jeweiligen Register-Files als mapping für die vier virtuellen Register verwendet.
Die Berechnung der Hamming-Distanzen fuer beide Versionen und die daraus ergebene Schaltaktivitätsminimierung kann in \ref{tab::hamming_distanz} nachvollzogen werden. Dabei wird auf Grund der Symmetrie nur Register-File 0 betrachtet.
Aus dem Minimalbeispiel ergibt sich ein unterschied der Hamming-Distanzen von sechs. Dies bedeutet,dass die alte Version sechs Schaltaktivitäten mehr aufweist und somit einen höheren Energie verbraucht verzeichnet. Weiter Untersuchungen zur Verlustleistungseinsparung werden im Evaluationskapitel \ref{chap:evaluation} beschrieben.

\begin{table}
	\centering
	$ \begin{array}{|c|c|}
	\hline
	alte Version & neue Version \\
	\hline
\text{V0R0 -->V0R0: } 00000 \text{ und } 00000 -> H= 0 &\text{V0R0-->V0R0: } 00000 \text{ und } 00000 -> H= 0\\
\text{V0R0 -->V0R2: } 00000 \text{ und } 00010 -> H= 1 &\text{V0R0-->V0R2: } 00000 \text{ und } 00010 -> H= 1 \\
\text{V0R2 -->V0R4: } 00010 \text{ und } 00100 -> H= 2 &\text{V0R2-->V0R4: } 00010 \text{ und } 00100 -> H= 2 \\
\text{V0R4-->V0R31: } 00100 \text{ und } 11111 -> H= 4 &\text{V0R0-->V0R5: } 00100 \text{ und } 00101 -> H= 1 \\
\text{V0R31-->V0R30: } 11111 \text{ und } 11110 -> H= 1 &\text{V0R5-->V0R1: } 00101 \text{ und } 00001 -> H= 1 \\
	\sum H=8 &\sum H=5 \\
	\hline
	\end{array} $
	\caption{Hamming-Distanz-Berechnung}
	\label{tab::hamming_distanz}
\end{table}

virtuelle Ports fuer X2 MOde


\section{Genetischer Algorithmus}
\label{sec:genetischerAlgorithmus}
Da nur Target-Adressen mit Hilfe von virtuellen Registern frei gewählt werden können, kann die Heuristik nur diese Optimieren, dennoch haben die Target-Register Einfluss auf die Source-Addressen und somit auch auf die Schaltaktivität. Auch wenn die Adresstoggel der Target-Regsiter minimiert wurde, kann es zu ineffizienten Adressnutzung der Source-Register kommen. Beim Allokieren der Register kann außerdem keine Aussage getroffen werden, ob und wie oft im Code Adressen verwendet werden. Aus diesem Grund wurde ein genetischer Algorithmus ausgewählt, mit welchem es möglich ist, das Toggeln sowohl in den Source-Adressen als auch in den Target-Adressen zu minimieren. Die Funktion und Bestandteile sollen nun erläutert werden.

\subsection{Gene und Chromosomen}
Die Gene des implementierten genetischen Algorithmus wurden so gewählt, dass diese die zu allokierenden Register repräsentieren. Hierbei wird für jedes Gen ein Register zugewiesen.
Alle $m$ Gene befinden sich in einem Chromosom zusammengefasst, $m$ ist hierbei die Anzahl der virtuellen Registern innerhalb einer SLM. Bei dem Gen, handelt es sich um zwei integer Werte $i$ und $j$. Hierbei ist $i$ das Register-File welches sich in dem Bereich von 0..1 bewegen kann, da es sich um eine Architektur mit zwei Issue-Slots handelt. $j$ weißt Werte zwischen 0..31 auf und repräsentiert die Register-Adresse. Beide Werte werden so verrechnet, dass eine eindeutige Register-ID hervor geht (siehe Gleichung \ref{eq:Gen}). Dazu wird $i$ mit der Anzahl an Register multipliziert und der Wert $j$ addiert. 
\begin{equation}
\text{Adress-ID }= (i* \text{Anzahl an Registern= 32}+ j)
\label{eq:Gen}
\end{equation}
 
 Innerhalb des Chromosoms befindet sich zusätzlich der interne Fitness-Wert, der Fitness-Wert für die Weitergabe an das Scheduling, der Ursprung des Chromosoms und die beiden Elternpaare. Die Fitness-Werte werden im weiteren Verlauf \ref{chap:Fitness-Funktion} näher beschrieben. Bei dem Ursprung handelt es sich um eine Angabe aus welchem Teil des genetischen Algorithmus das Chromosom entstanden ist. Es bestehen vier Möglichkeiten: letzte Population, Crossover, Mutation oder Random. Die Elternpaare geben zusätzlich eine Information über die Eltern des Chromosoms. Durch diese Parameter kann genetischen Algorithmus besser analysiert werden und die Fortpflanzung der Gene nachvollzogen werden. Außerdem kann evaluiert werden ob beispielsweise die Mutation eine Verbesserung findet. 
  
 
\subsection{Aufbau}
\label{chap:aufbau}
Zu Beginn des genetischen Algorithmus, muss eine Startpopulation festgelegt werden. Eine Population ist hierbei ein Satz an zufällig gewählter Register-Allokationen auch Chromosom genannt. Dabei wird darauf geachtet, dass keine Adressen zugewiesen werden die bereits von physikalischen Registern blockiert sind.
Um die einzelnen Mitglieder einer Population unterscheiden zu können, muss für jedes Chromosom ein Fitness-Wert \ref{chap:Fitness-Funktion} ermittelt werden. Mithilfe dieses Merkmales wird die Population im Anschluss sortiert. 
Mit den generierten Register-Allokations-Ansätzen beginnt nun Fortpflanzen der Population. In einer Schleife werden zuerst die beiden besten Chromosomen der letzten Generation in eine Neue abgespeichert. Anschließend werden die restlichen Mitglieder der Population durch Crossover, Mutation und neu erzeugte Chromsomen aufgefüllt. Zum Ende der Schleife wird die neu errungenen Chromosomen anhand des Fitness-Werts sortiert. Diese geschieht so lange, bis eine maximale Anzahl an Durchlaufen erreicht wurde. Sobald eine Verbesserung durch die Fortpflanzung und Veränderung der Chromosom gefunden wurde, beginnt die Schleife von Neuem. Der Algorithmus terminiert dementsprechend immer wenn nach einer gewissen Anzahl an Durchlaufen keine Verbesserung des Fitness-Werts gefunden werden kann. Nach dem Beenden der Schleife wird die beste Population ermittelt und für das Mapping von virtuellen zu physikalischen Registern verwendet.

\begin{algorithm}[H]
	\begin{algorithmic}[1]
	\STATE {Startpopulation \(P_{S}\) ermitteln}
	\STATE {Population \(P_{S}\) sortieren}
	\WHILE{Anzahl der Durchläufe < maximale Anzahl an Durchläufen}
		\STATE {neue Population \(P_{N}\) erstellen}
		\STATE {zwei Besten Chromosomen der alten Population \(P_{S}\)  in die neue  Population \(P_{N}\) übernehmen}
%		
		
		\FOR{Anzahl an Fortpflanzungen>0} \STATE{
			\textbf {Auswahl:} Auswahl der Elternpaare \(E_{1}\)+\(E_{2}\) }				\STATE{ mit Roulette-Wheel-Selection aus \(P_{N}\)}
		 	\STATE{\textbf {Crossover:} Fortpflanzung der Eltern \(E_{1}\)+\(E_{2}\) = \(C\)}
		 	\STATE{\textbf {Mutation:} Mutieren des neuen Chromosoms \(C\)}
		 	\STATE{\textbf {Einfügen:} Einfügen von \(C\) in die Population \(P_{N}\)}
		 
	 	\ENDFOR
	  	\FOR{Anzahl an neuen Chromosomen}
	 		\STATE{\textbf {Create:} Erstellen von neuem Chromosom \(C_{N}\)}
	 		\STATE{\textbf {Einfügen:} Einfügen von \(C_{N}\) in die Population\(P_{N}\)}
		\ENDFOR
		\STATE {Population \(P_{N}\) sortieren}
		\IF{Verbessung zu \(P_{S}\) gefunden}
			\STATE {Durchläufe =0}
		\ELSE
			\STATE {Durchläufe + 1}
		\ENDIF
	\ENDWHILE
	\caption{Pseudocode genetischer Algorithmus}
\end{algorithmic}
\end{algorithm}

\subsection{Auswahl der Elternpaare}

Wie bereits im Grundlagenteil \ref{chap:grundlagen_cossover} skizziert handelt es sich beim Crossover um das Fortpflanzen der Chromosomen.
Das Fortpflanzungspaar wird hierbei mittels dem bereits erwähnten Tournament Selection \ref{chap:grundlagen_cossover} ausgewählt. Die Literatur \cite{zhong2005comparison} hat gezeigt, dass das Tournament-Selection-Verfahren sehr viel schneller gegen einen Wert konvergiert als das Roulette-Wheele-Selection-Verfahren. Aus diesem Grund wird für die Selektion der Elternpaare das Tournament-Select-Verfahren verwendet. Dieses wählt aus der Population $P_S$, $n$ zufällige Chromosomen aus. Die so gewählten Chromosomen treten in einem \"Tunier(Tournament)\" gegeneinander an. Dabei werden die einzelnen Chromosom mithilfe des Fitness-Wertes verglichen und das Chromosom welches die besten Fitness aufweist ausgewählt. Mit Hilfe dieses Verfahren werden zwei individuelle Chromosomen aus der alten Population $P_S$ ausgewählt.
 
\subsection{Fortpflanzung der Chromosomen}
Dieses Paar wird daraufhin an einer zufällig gewählten Stelle getrennt und die entstandenen Teile zusammengeführt. Daraus entsteht ein neues Kind-Chromosom. Ein Beispiel einer Fortpflanzung ist in Abbildung \ref{fig:crossover} dargestellt.
\begin{figure}[htbp] 
	\centering
	\includesvg[width=\textwidth]{Crossover}
	\caption{Fortpflanzung mittels Crossover}
	\label{fig:crossover}
\end{figure}
Nehmen wir an $m$ ist die Anzahl an Genen in einem Chromosom und $r \in \mathbb{N}$ ist eine Zufallszahl, so berechnet sich die Stellen $x$ an der ein Wechsel der Eltern-Gene auftritt mit: $x= min(x+r mod(m+1),m)$. Nun werden $x$ des ersten Elternpaars $E_1$ in die des Kind-Chromosoms $C$ kopiert. Die berechnete Stelle wird auch Crosspoint genannt. Im Anschluss an das erste Crossover wird die Berechnung des Crosspoints $x$ wiederholt und die entsprechenden Gene aus $E_2$ kopiert. Dieses Verfahren wird so lange wiederholt, bis $C$ mit Genen aus $E_1$ und $E_2$ befüllt ist. Durch dieses Vorgehen, kann es zu mehr als einem Crosspoint kommen, welches laut Literatur ein besseres Ergebnis liefert.\cite{reeves2003genetic} \\
Nach dem durch das Crossover, ein neues Chromosom $C$, mit den Genen aus zweier Eltern $E_1$ und $E_2$ der alten Generation entstanden ist, wird dieses zusätzlich mutiert. Dabei wird jedes Gen in $C$ mit einer Wahrscheinlichkeit von 1\% modifiziert. Die Änderung bezieht sich hierbei auf das Register-File und die Register-Adresse.Dabei werden die Werte zufällig zugewiesen, die Register-Adresse kann hierbei in einem Bereich von 0..31 und das Register-File zwischen 0..1 liegen.\\
Das so entstandene neue Chromosom $C$ wird nun zur neuen Population $P_N$ hinzugefügt. Der Fortpflanzungsprozess aus Crossover und Mutation wird so lange durchgeführt bis die neue Population $P_N$ mit der vorbestimmten Anzahl an Chromosomen aus der Fortpflanzung sowie mutierten befüllt wurde.\\
Im Anschluss werden neue Chromosomen erstellt und er Population hinzugefügt. Dies stellt zusätzlich zur Mutation sicher, dass der Algorithmus nicht an einem lokalen Minimum gefangen bleibt. 

\subsection{Fitness-Funktion}
\label{chap:Fitness-Funktion}
Der ideal Fall für die Fitness-Funktion wäre eine Einbindung eines Power-Analyse-Tools in den Algorithmus. Dies würde bedeuten, dass für jede Berechnung der Fitness das Tool ausgeführt werden müsste und die Verlustleistung berechnet. Dadurch wäre eine exakte Aussage über die Leistung möglich und der genetische Algorithmus würde diese minimieren. Da das Analyse-Tool jedoch sehr lange Ausführungszeiten aufweist ist dies nicht möglich. Aus diesem Grund muss eine Konvergenz zur Power gefunden werden.\\
Wie in Kapitel \ref{chap:ansatz} bereits beschrieben wurde die Hamming-Distanz als Annäherungsparameter verwendet. Da die Registerallokation dem Scheduling untergeordnet ist und von diesem aufgerufen wird, muss die Fitnessfunktion an die des Scheduling angepasst werden. Aus diesem Grund wurde zum einfacheren Vorgehen zwei Verschiedene Fitness-Funktionen implementiert. Zum einen der Wert der an die übergeordnete Funktion (das Scheduling) weitergegeben wird und zum Andern eine interne Fitness um die Ergebnisse zu evaluieren.
Der Übergabewert besteht aus der Anzahl der Register die nicht im Register-File allokiert werden konnten. Bei der internen Fitness-Funktion bestehen mehrere Ansatzmöglichkeiten.\\
Das sortieren der Population wird mit beiden Fitness-Funktionen realisiert. Da die Register-Allokation nur dann erfolgreich war, wenn alle Register allokiert werden konnten, ist dies der erste Schlüssel nach dem die Population sortiert wird. Als zweiter Sortierschlüssel wird die interne Fitness-Funktion verwendet.
Besteht der Fall, dass ein Register mit dem zufällig gewählten Adressen nicht zugewiesen werden kann, so wird der Zähler für nicht allokierbare Register erhöht. Zusätzlich wird die Fitness berechnet und mit einem Offset versehen. Somit können Chromosomen mit der selben Anzahl an nicht allokierbaren Registern untereinander verglichen werden. Dies ist nötig da sonst das Crossover zu keinen Verbesserung der Chromosomen kommen kann. Außerdem ist durch den Einsatz eines großen Offsets eine Abgrenzung zu den geglückten Zuweisungen möglich. Der Offset wurde hierfür  so gewählt, dass selbst bei dem Worst-Case-Szenario, wenn die Hamming-Distanz in jedem Schritt maximal wäre, eine klare Separierung möglich ist. Dazu wurde die Anzahl der virtuellen Registern in einer SLM mit 1000 multipliziert.
Im Folgenden wird auf die einzelnen Ansätze der Fitness-Funktion eingegangen:

\begin{itemize}
	\item Hamming-Distanz\\
		Als erste Variante wurde die Summe der Hamming-Distanzen als Fitnesswert verwendet. Hierzu wurden die Hammingdistanzen der Target- sowie Source-Register ermittelt. Wie im Falle der Heueristik mussten die verschiedenen Issue-Slots berücksichtigt werden. Da jeden Adress-Dekoder zwei Source-Adressen aufweist, gibt es in diesem Fall vier Möglichkeiten der Allokation.
	\item gewichtete Hamming-Distanz\\
		blabla
	\item Lastkapazitaet\\
		blabla
	\item Hamming-Distanz und Lastkapazität\\
		blabla
%	\item Adresssumme\\
		%blabla
\end{itemize}

Schaubild Fitness-Funktion -> Power
 
\subsection{Parameter}
\label{cap:parameter}
Es bestehen eine Vielzahl an Parametern und Einstellungsmöglichkeiten bei der Verwendung eines genetischen Algorithmus, dabei haben einige einen erheblichen Einfluss auf die Funktion und das Ergebnis. Da die Parameter sehr von der Problemstellung abhängen, kann nur begrenzt auf Literatur zurück gegriffen werden. Jedoch wurden eine Größenordnung und erste Richtwerte aus der Literatur \cite{grefenstette1986optimization} und dem genetischen Scheduling entnommen. Im Folgenden sollen die Parameter und deren Funktion sowie Einfluss auf das Ergebnis erläutert werden.

\begin{itemize}
	\item Anzahl der Durchläufe\\
		Die Anzahl der Durchläufe wurde experimentell ermittelt und ist für jede Problemstellung unterschiedliche. Hierfür wurde der beste Fitness-Wert über der Anzahl an Durchläufen in einem Diagramm aufgetragen. An der Stelle an der der Algorithmus keine signifikanten Verbesserungen findet wurde der Abstand zwischen den zuletzt aufgetretenen Optimierungen ermittelt. Dieser wurde im Anschluss als maximale Durchlaufzahl verwendet.
		
		Schaubild Fitness über Durchläufe 
	\item Größe der Population $N$\\
		Die Populationsgröße $N = 40$ wurde aus der Literatur \cite{grefenstette1986optimization} und Erfahrungswerte aus dem genetischen Instruktion-Scheduling entnommen.  
	
	\item Mutations-Wahrscheinlichkeit $w_M$\\
		Die Mutation ist dem Crossover untergeordnet und hebt die Flexibilität des Algorithmus. Dabei sind sehr kleine Werte zu wählen. Die Wahl fiel hierbei auf eine Wahrscheinlichkeit $w_M$ von einem Prozent. Dies bedeutet das $w_M * N * n$ Gene pro Population mutiert werden. Hierbei ist $n$ die Anzahl der Gene in einem Chromosom. Ist $w_M$ zu hoch gewählt, handelt es sich um eine Zufallssuche.
		
	\item Startwerte\\
		Die Startwerte haben einen erheblichen Einfluss auf die Laufzeit und Konvergenz des genetischen Algorithmus. Da die Gene zufallsgesteuert sind kann auf diesen Parameter nur bedingt Einfluss genommen werden. Durch das Vorschalten der Heuristik und damit der Berechnung einer gültigen Register-Allokation, kann der Algorithmus bereits an einem besseren Ausgangsposition starten.(siehe Kapitel \ref{sub:startpop})
	\item Fitnessfunktion\\
		Den größten Einfluss auf die Funktion des Algorithmus hat jedoch die Fitness-Funktion. Diese ist ausschlaggebend für die Anpassung an das gegebene Problem und den Endwert des Algorithmus. Wird die Fitness-Funktion falsch gewählt, ist das Ergebnis des genetischen Algorithmus nicht aussagekräftig. 
\end{itemize}
 
\subsection{Startpopulation}
\label{sub:startpop}
Die Startpopulation besteht im Normalfall wie in Kapitel \ref{chap:aufbau} aus zufällig gewählten Genen. Hierbei werden nur Register verwendet die nicht bereits von physikalische Registern blockiert sind.
Um eine schnellere Konvergenz zu einem globalen Minimum zu gewährleisteten wird ein Chromosom der Startpopulation durch ein Heuristik-Algorithmus bestimmt. Dadurch wird bereits an einem lokalen Minimum gestartet und der genetische Algorithmus muss dieses nicht selbst finden. 
Nachteilig ist hierbei jedoch das durch ungeeignete Parameter der Algorithmus bei dem bereits sehr starken lokalen Minimum stagniert und keine Verbesserung mehr finden kann. Aus diesem Grund sollte beim Start mit der Heuristik eine höhere Mutations-Wahrscheinlichkeit eingesetzt werden um dieses Minimum dennoch zu überwinden.

\subsection{dynamische Anpassung}
Um eine schnellere Konvergenz gegen ein globales Minimum zu erreichen, wurde eine dynamische Anpassung der Parameter implementiert. Dies ist sinnvoll, da zu beginn die Suche auf einen großen Bereich ausgelegt werden muss, jedoch im weiteren Verlauf diese Parameter eventuell nicht mehr die schnellste Option darstellen. Aus diesem Grund wurde die Mutationsrate während der Berechnung angepasst. Zu beginn des Algorithmus werden viele bessere Lösungen als die der Startpopulation gefunden. AUs diesem Grund wird mit einer sehr kleinen Mutationsrate gestartet. Sobald für eine gewisse Anzahl an Durchläufen keine Verbesserung gefunden wurde, sprich der Algorithmus gegen ein lokales Minimum tendiert, wird die Mutations-Wahrscheinlichkeit erhöht. Dies hat zur folge, dass die Suche nach geeigneten Genen breiter wird bzw. mehr Gene mutiert werden. Dadurch steigt die Wahrscheinlichkeit ein lokales Minimum zu überwinden an. Wird eine Verbesserung gefunden, wird die Mutationsrate auf den Startwert zurückgesetzt. Anderenfalls wird die Wahrscheinlichkeit weiter minimiert. Damit wird aus einer gezielten Suche wie in Abschnitt \ref{cap:parameter} erläutert eine Zufallssuche.
Das untenstehende Schaubild zeigt die Optimierung, die durch diese Verfahren erreicht wurde. Hierbei wurden beide genetischen Algorithmen mit dem selben Seed/Startwert ausgeführt um die Versionen vergleichen zu können.

Schaubild Dynamisch->statischer GA
 
\newpage
\subsection{Algorithmus Modi}
\begin{itemize}
	 \setlength{\itemsep}{-6pt}
	\item-E -> Register Allokations-Modus Heuristik
			\begin{itemize}
				\setlength{\itemsep}{-4pt}
				\item Alt
				\item Neu
			\end{itemize}
	\item-M -> Register Allokations-Modus Genetischer Algorithmus
		\begin{itemize}
			\setlength{\itemsep}{-4pt}
			\item Hamming-Distanz
			\item Load
			\item Hamming*Load
		\end{itemize}
	\item-G -> Auswahl der Algorithmen
			\begin{itemize}
				\setlength{\itemsep}{-4pt}
				\item Genetischer Algorithmus
				\item Heuristik
			\end{itemize}
	\item-D -> dynamischer Genetischer Algorithmus
	\item-H -> Heuristik als Startgen
	\item-U -> Mutation-Modus
				\begin{itemize}
					\setlength{\itemsep}{-4pt}
					\item Mutation auf Crossover-Gen
					\item Mutation auf Zufalls-Gen
				\end{itemize}
	\item-o -> Optimierungsgrad
			\begin{itemize}
				\setlength{\itemsep}{-4pt}
				\item o1 List Scheduling
				\item o2 Genetisches Scheduling 20 Gene
				\item o3 Genetisches Scheduling 100 Gene
				\item o4 Genetisches Scheduling 200 Gene
			\end{itemize}
	
\end{itemize}


	%Tabelle mit allen Modis 
	\begin{table}[htp]
		\centering
		\begin{tabular}{ccccccccc}
			\multicolumn{2}{l}{}                 & \multicolumn{7}{|l}{Modi}                                                               \\ 
			\multicolumn{2}{c}{Algorithmus-Konfig.}  & \multicolumn{1}{|c}{-E} & \multicolumn{1}{c}{-M} & \multicolumn{1}{c}{-G} & \multicolumn{1}{c}{-D} & \multicolumn{1}{c}{-H} & \multicolumn{1}{c}{-U} & \multicolumn{1}{c}{-o} \\ 
			\hline
		\multicolumn{2}{l}{Heuristik alt}  & \multicolumn{1}{|c}{0} & \multicolumn{1}{c}{1} & \multicolumn{1}{c}{0} & \multicolumn{1}{c}{X} & \multicolumn{1}{c}{X} & \multicolumn{1}{c}{X} & \multicolumn{1}{c}{1} \\
		\multicolumn{2}{l}{Heuristik neu}  & \multicolumn{1}{|c}{1} & \multicolumn{1}{c}{1} & \multicolumn{1}{c}{0} & \multicolumn{1}{c}{X} & \multicolumn{1}{c}{X} & \multicolumn{1}{c}{X} & \multicolumn{1}{c}{1} \\ 
		\multicolumn{2}{l}{Genetischer Algo. Hamming-Fitness}  & \multicolumn{1}{|c}{1} & \multicolumn{1}{c}{1} & \multicolumn{1}{c}{1} & \multicolumn{1}{c}{0} & \multicolumn{1}{c}{0} & \multicolumn{1}{c}{0} & \multicolumn{1}{c}{2-4} \\
			\multicolumn{2}{r}{Heuristik als Startpop.}  & \multicolumn{1}{|c}{1} & \multicolumn{1}{c}{1} & \multicolumn{1}{c}{1} & \multicolumn{1}{c}{0} & \multicolumn{1}{c}{1} & \multicolumn{1}{c}{0} & \multicolumn{1}{c}{2-4} \\
			\multicolumn{2}{r}{dynamischer Genetischer Algo.}  & \multicolumn{1}{|c}{1} & \multicolumn{1}{c}{1} & \multicolumn{1}{c}{1} & \multicolumn{1}{c}{1} & \multicolumn{1}{c}{0} & \multicolumn{1}{c}{0} & \multicolumn{1}{c}{2-4} \\
		\multicolumn{2}{l}{Genetischer Algo. Load-Fitness}  & \multicolumn{1}{|c}{1} & \multicolumn{1}{c}{2} & \multicolumn{1}{c}{1} & \multicolumn{1}{c}{0} & \multicolumn{1}{c}{0} & \multicolumn{1}{c}{0} & \multicolumn{1}{c}{2-4} \\ 
		\multicolumn{2}{l}{Genetischer Algo. Hamming*Load-Fitness}  & \multicolumn{1}{|c}{1} & \multicolumn{1}{c}{3} & \multicolumn{1}{c}{1} & \multicolumn{1}{c}{0} & \multicolumn{1}{c}{0} & \multicolumn{1}{c}{0} & \multicolumn{1}{c}{2-4}                     
		\end{tabular}
		\newline
		\caption{Algorithmus-Konfiguration}
	\end{table}

