\chapter{Schlussfolgerung}
\label{chap:schlussfolgerung}
Das Ziel dieser Arbeit war es, durch einen genetischen Optimierungsalgorithmus eine Register-Allokation zu finden, welche die Verlustleistung des KAVUAKA-Prozessors minimiert. Es wurde zu Beginn die These aufgestellt, dass die Verlustleistung proportional mit der Schaltaktivität der Register-Adressleitungen steigt. Durch einen Vergleich der Best- und Worst-Case-Adressierung, wurde ein Optimierungspotential von 7,87\% für die Verlustleistung des gesamten Prozessors und 18,33\% für die Schaltleistung des Register-Files ermittelt. Dadurch konnte ein klarer Zusammenhang der Schaltaktivität und der Verlustleistung festgestellt werden. Um diesen Ansatz in der Register-Allokation umzusetzen, wurde zu Beginn ein Modell gewählt, dass den Freiheitsgrad von virtuellen Registern ausnutzt und diese so allokiert, dass die Schaltaktivität der Register-Adressierung minimiert wird. Mittels einer neu implementierten Heuristik wurde so mithilfe der Hamming-Distanz die Schaltaktivität der Target-Port-Adressen minimiert. Dabei wurden in den programmierten synthetischen Tests eine Verlustleistungseinsparung von 1,82\% für die Gesamtverlustleistung gegenüber der alten Implementierung ohne Schaltaktivitätsoptimierung erzielt.
Um die Schaltleistungen noch weiter zu minimieren und ebenfalls die Source-Port-Adressen in die Optimierung mit einzubinden, wurde ein genetischer Optimierungsalgorithmus implementiert. Dieser optimiert anhand der Hamming-Distanz, gepaart mit der Lastkapazität als Fitness-Wert, die Schaltaktivität im Multishared Register-File. Dabei hat die Evaluation mittels synthetischen Tests gezeigt, dass die Parameter des Algorithmus an das jeweilige Problem angepasst werden müssen und nicht pauschalisiert werden können. Eine Startpopulationsberechnung durch eine Heuristik und eine dynamische Anpassung der Mutationswahrscheinlichkeit ist jedoch für alle Problemstellungen sinnvoll.
Durch den Einsatz des genetischen Algorithmus kann die Hamming-Distanz der Adressen minimiert werden, welches eine Minimierung der Schaltleistung der Adress-Ports von durchschnittlich 41,57\% mit sich führt.  
Die Evaluation hat gezeigt, dass dadurch eine geeignete Adressierung gefunden werden kann, die die dynamische Verlustleistung insbesondere in den Register-Files minimiert. Dabei liegt die Einsparung gegenüber der alten Implementierung bei 1,62\% und konnte keine weitere Verbesserung zur neuen Heuristik generieren.
Für den realen Testfälle der Emulated-Floating-Point FFT sind Einsparungen von 0,41\% zu erzielen. Dabei ist zu beachten, dass es zu Schaltaktivitäten der Daten auf Grund des Verhaltens der Adressdecoder kommt und diese eine Verschlechterung der Verlustleistungsergebnisse in den realen Testfällen erzeugen.
Die erzielten Einsparungen mögen nicht hoch erscheinen, werden jedoch nur durch eine Veränderung der Software erreicht. Dadurch muss kein Abtausch zwischen Energieeffizienz und Chipfläche stattfinden, welches einen deutlichen Vorteil gegenüber anderen Ansätzen darstellt.
Außerdem wurde der Einfluss der Lastkapazität auf die Verlustleistung festgestellt und aufgezeigt. Durch eine Berücksichtigung dieser, kann die Leistung weiter optimiert werden.
Dadurch, dass die Register-Allokation dem Scheduling untergeordnet ist, kann es zu sehr langen Berechnungszeiten kommen. Dies ist insbesondere der Fall, wenn für das Scheduling und die Register-Allokation ein genetischer Algorithmus gewählt wurde. Außerdem steht die Berechnungsdauer mit der Anzahl an virtuellen Registern bei der genetischen Register-Allokation in Korrelation. Da jedoch für die meisten Programme ein List-Scheduling ausreicht, kann der genetische Algorithmus für die Register-Allokation eingesetzt werden.\\
Weitere ausgiebige Tests und Optimierungen der Parameter des genetischen Algorithmus können in Zukunft das Verhalten der Verlustleitung weiter an das Optimierungspotential angleichen und so den Energiebedarf weiter senken. Außerdem könnte durch den Einsatz von Clock-Gating eine weitere Funktion eingebaut werden, die es ermöglicht, die Schaltaktivität und die damit verbundene Verlustleistung in ungenutzten Teilen des Prozessors zu senken. Durch den Einsatz von Dummy-Registern kann zudem verhindert werden, dass Variablen mit geringer Lebensdauer an das Register-File zurückgeschrieben werden und dort Verlustleistungen verursachen. Außerdem ist dieser Ansatz auf andere DSPs übertragbar und kann auch dort ohne Performance-Verlust oder Hardwareanpassung die Verlustleistung minimieren.\\
Alles in allem wurde gezeigt, dass durch eine Anpassung des Schedulers die Verlustleistung um nahezu 1\% gesenkt werden kann. Es wird Hörgerätnutzern dadurch ermöglicht längere Zeit besser zu hören und ihre Lebensqualität zu erhöhen.\\
%
%Außerdem wurde eine weitere Pipeline-Stufe in dem Prozessor eingebaut. Diese verhindert, dass Glitches durch die Berechnung der Adressen an die Register-Files weitergeleitet werden und dort Schaltleistung verursacht.


