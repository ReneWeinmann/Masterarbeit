\chapter{Schlussfolgerung}
\label{chap:schlussfolgerung}
Das Ziel dieser Arbeit war es, durch einen genetischen Optimierungsalgorithmus eine Register-Allokation zu finden, welche die Verlustleistung des KAVUAKA-Prozessors minimiert. Dabei wurde zu Beginn der Ansatz gewählt die Register so zu allokieren, dass die Schaltaktivität der Register-Adressierung minimiert wird. Hierzu wurde ein genetische Optimierungsalgorithmus implementiert, welcher anhand der Hamming-Distanz gepaart mit der Lastkapazität als Fitness-Wert die Schaltaktivität im Multishared Register-File optimiert. Dabei hat die Evaluation gezeigt, dass die Parameter des Algorithmus an das jeweilige Problem angepasst werden muss und nicht pauschalisiert werden kann. Eine Startpopulationsberechnung durch eine Heuristik und dynamische Anpassung der Mutationswahrscheinlichkeit ist jedoch für alle Problemstellungen sinnvoll.
Dadurch, dass die Register-Allokation dem Scheduling untergeordnet ist, kann es zu sehr langen Berechnungszeiten kommen. Dies ist insbesondere der Fall, wenn für das Scheduling und die Register-Allokation ein genetischer Algorithmus gewählt wurde. Außerdem steigt die Berechnungsdauer mit der Anzahl an virtuellen Registern bei der genetischen Register-Allokation an. Da jedoch für die meisten Programme ein List-Scheduling ausreicht, kann der genetische Algorithmus für die Register-Allokation immer eingesetzt werden.
Durch den Einsatz des genetischen Algorithmus kann die Hamming-Distanz der Adressen minimiert werde, welches eine Minimierung der Schaltleistung der Adress-Ports mit sich führt.  
Die Evaluation hat gezeigt, dass dadurch eine geeignete Adressierung gefunden werden kann, die die dynamische Verlustleistung insbesondere in den Register-Files minimiert werden kann. Das Optimierungspotential beträgt hierbei im Idealfall 7,87\% für die Verlustleistung des gesamten Prozessors und 38,65\% für die Schaltleistung.  Weitere Untersuchungen haben gezeigt dass Werte im Bereich von 1\% für die Gesamtverlustleistung wahrscheinlicher sind. Für die realen Testfälle des Beamforming und emulated-floation-point sind Einsparungen von XXX \% zu erzielen. Dabei ist zu beachten, dass es zu Schaltaktivitäten der Daten auf Grund des Verhaltens der Adressdecoder kommt und diese eine Verschlechterung der Verlustleistungsergebnisse erzeugt. Außerdem wurde der Einfluss der Lastkapazität auf die Verlustleistung bewiesen und aufgezeigt. Durch eine Berücksichtigung dieser kann die Leistung weiter optimiert werden.
Außerdem wurde eine weitere Pipeline-Stufe in dem Prozessor eingebaut. Diese verhindert, dass Glitches durch die Berechnung der Adressen an die Register-Files weitergeleitet werden und dort Schaltleistung verursacht.


Durch diese Arbeit wurde gezeigt, dass eine Optimierung der Register-Allokation mithilfe eines genetischen Optimierungsalgorithmus sinnvoll ist um die Verlustleistung eines Prozessors zu minimieren. Dabei sollte die Fitness des Algorithmus so gewählt werden 


