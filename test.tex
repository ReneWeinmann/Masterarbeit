\documentclass{article}
% Damit die Verwendung der deutschen Sprache nicht ganz so umst\"andlich wird,
% sollte man die folgenden Pakete einbinden:

% FUER INKSCAPE IMPORT
\usepackage[pdftex]{graphicx}
\usepackage{calc}

\usepackage{svg}
\setsvg{inkscape=inkscape -z -D,svgpath=fig/}
%%%%%%%%%%%%%%%%%%%%%%%%%%%%%%%%%%%%%%%%%%%%%%%%%%%%%%%%%%%


\usepackage[latin1]{inputenc}% erm\"oglich die direkte Eingabe der Umlaute 
\usepackage[T1]{fontenc} % das Trennen der Umlaute
\usepackage{ngerman} % hiermit werden deutsche Bezeichnungen genutzt und 
                     % die W\"orter werden anhand der neue Rechtschreibung 
		     % automatisch getrennt.  
\usepackage{color}
\title{Ein kleine \LaTeX{} Article Vorlage\thanks{Wem auch immer
	 der Dank gelten mag\ldots}}
\author{Ihr Name  \\
	Ihr Unternehmen / Universit\"at  \\
	Teststra\ss e -99 \\
	0123456 Testhausen \\
	\and 
	Der Andere  \\
	Sein Unternehmen / Universit\"at \\
	Musterstra\ss e 00 \\
	6543210 Musterdorf \\
	}

\date{\today}
% Hinweis: \title{um was auch immer es geht}, \author{wer es auch immer 
% geschrieben hat} und  \date{wann auch immer das war} k\"onnen vor 
% oder nach dem  Kommando \begin{document} stehen 
% Aber der \maketitle Befehl mu\ss{} nach dem \begin{document} Kommando stehen! 
\begin{document}

\maketitle


\begin{abstract}
Eine kurze Zusammenfassung um was es in der vorliegenden Arbeit \"uberhaupt
so geht \ldots
\end{abstract}

\section{Einf\"uhrung}
Ziel der Arbeit ist es m\"oglichsten vielen oder wenn m\"oglichen es allen 
zu erm\"oglichen, Dokumente mit \LaTeX{} zu erstellen!




\begin{figure}
\centering
\includesvg[width=1.0\textwidth]{drawing}
\end{figure}

\section{Dokumentenklassen} \label{documentclasses}

\begin{itemize}
\item article
\item book 
\item report 
\item letter 
\end{itemize}


\begin{enumerate}
\item article
\item book 
\item report 
\item letter 
\end{enumerate}

\begin{description}
\item[article\label{article}]{Article ist \ldots}
\item[book\label{book}]{Die book Klasse ist \ldots}
\item[report\label{report}]{Die Klasse report erm\"oglicht es  \ldots}
\item[letter\label{letter}]{Wenn man einen Breif schreiben sollte man eine 
	andere Klasse nutzen, da diese f\"ur ein anderes als das deutsche 
	Briefformat ausgelegt ist.}
\end{description}


\section{Fazit}\label{conclusions}
Nach langer Suche hat sich herausgestellt, dass es kein l\"angeres 
\LaTeX{} Beispiel, als das von \cite{doe} geschriebene gibt. 

\begin{thebibliography}{9}
\bibitem[Doe]{doe} \emph{Erstes und letztes \LaTeX{} Beispiel.},
John Doe 50 v.Chr.  
\end{thebibliography}

\end{document}
