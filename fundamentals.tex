% !TeX spellcheck = de_DE
\chapter{Grundlagen digitaler Signalverarbeitung in Hörgeräten}
\label{chap:grundlagen}
In diesem Kapitel wird kurz auf die Grundlagen und Begriffe von Signalprozessoren anhand der zugrundeliegenden Architektur eingegangen. Im Anschluss werden einzelne, für diese Arbeit relevante Bestandteile genauer beschrieben. Daraufhin wird ein kurzer Einblick in verwandte Ansätze gegeben werden, um im weiteren Verlauf die Funktion des Scheduling und das Prinzip der Verlustleistung zu beschreiben. Das Ende des Kapitels gibt eine grobe Übersicht über genetische Algorithmen.

%\section{FPGA}
%Bei einem FPGA (Field Programmable Gate Array) handelt es sich um ein frei programmierbaren integrierten Schaltkreis, welcher mithilfe von Look-Up-Tabellen sehr effizient Algorithmen abbildet. Um eine gewünschte Funktion auf dem FPGA zu implementieren ist es notwendig die Look-Up-Tabellen mit einer Wahrheitstabellen zu befüllen, welche diese Funktion abbildet. Dadurch kann ein FPGA jegliche erdenkliche Aufgabe erfüllen. Um die beschriebenen Blöcke mit dem IO-Ports des Chips zu verbinden ist es notwendig eine Verbindung mithilfe von Rooting-Switches und Connection-Boxen zu erstellen. Damit der Chip konfiguriert werden kann, muss die Funktion mithilfe einer HDL(Hardware Description Language) beschrieben werden.\cite{farooq2012fpga} Vorteil dieser Implementierung ist, dass eine echte Parallelität erreicht werden kann, so dass mehrere Blöcke zeitgleich ausführbar sind. In dieser Arbeit wird ein FPGA als Entwicklungsumgebung verwendet, um den zu entwickelnden ASIP bestmöglich abzubilden und dabei möglichst flexibel zu bleiben.


%\section{ASIP}
%ASIP ist die Abkürzung für Application Specific Instruction Set Processor. Die Architektur dieses Prozessors ist anwendungsspezifisch optimiert. Hierbei sind ASIPs meist auf eine gewisse Funktion optimiert. Durch diese Spezialisierung, ist es möglich Einsparungen hinsichtlich Siliziumfläche und Energieverbrauch gegenüber herkömmlichen Prozessoren zu erreichen. Der in dieser Arbeit verwendete ASIP ist eine vereinfachte Variante eines MIPS-Prozessors. MIPS oder auch \glqq Microprocessor without interlocked pipeline stages\grqq{} ist eine Befehlssatzarchitektur bei der der Entwickler/Compiler, Datenabhängigkeiten der Befehle erkennen und auflösen muss.

%\section{SIMD}
%\label{sec:SIMD}
%Der Begriff SIMD steht für Single Instruction Multiple Data und ist eine Datenverarbeitungseigenschaft von Prozessoren, die meist in der Signalverarbeitung zum Einsatz kommt. Der Vorteil einer solchen Implementierung ist, dass eine Instruktion in mehreren ALUs parallel, mit unterschiedlichen Daten ausgeführt werden kann. Somit ist nur eine Instruktion für die Berechnung vieler Operationen nötig, dadurch ist eine erhebliche Performanz-Steigerung erreichbar.\cite[Seite 249]{wust2010mikroprozessortechnik}

%\subsection{VLIW}
%\label{sec:VLIW}
%VLIW ist eine Eigenschaft von Mikroprozessor-Architekturen. VLIW steht hierbei für Very Long Instruction Word und bedeutet soviel wie sehr langes Instruktions-Wort. Das Ziel dieser Struktur ist eine schnelle Abarbeitung des Befehlsatzes, wobei hierbei einige Befehle parallel ausgeführt werden können. Um dies zu ermöglichen sind mehrere Instruktions-Dekoder vonnöten. Eine VLIW-Architekur geht meist mit Pipelining einher. 

%\subsection{Pipelining}
%Wie das Wort Pipelining schon besagt, handelt es sich hierbei um eine Befehlsabarbeitung am Fließband(Pipeline). Wurde ein Befehl in der ersten Pipelinestufe abgearbeitet kann dieser an die nächste Stufe weitergeleitet werden und der nachfolgende Befehl führt die erste Stufe aus. Die Zeit bis ein Befehl alle Piplinestufen passiert hat nennt man Latenzzeit und dauert genau so viele Takte wie die Pipeline Stufen hat.(siehe Abbildung \ref{fig:pipeline}) Diese Verzoegerung macht ist jedoch nur beim Befüllen der Pipeline bemerktbar, im Idealfall kann im Anschluss mit jedem Takt ein Befehl beendet werden. Somit ist eine parallele Verarbeitung mehrerer unterschiedlicher Befehle möglich. Dabei ist darauf zu achten, dass die Taktung der Pipelinestufen sich an der Stufe orientiert, welcher die längste Ausführungsdauer aufweist. \cite[Seite 204]{wust2010mikroprozessortechnik}
%\begin{scriptsize}
%	\begin{figure}[htbp] 
%		\centering
%		\includesvg[width=0.70\textwidth]{pipeline}
%		\caption{Pipeline-Prinzip}
%		\label{fig:pipeline}
%	\end{figure}
%\end{scriptsize}

\section{Aufbau der Architektur}
\label{chap:architecture_overview}
Bei dem verwendeten Prozessor handelt es sich um einen MIPS-Prozessor, der Teil des RAPANUI Projektes ist. Dieser ist als ASIP-Prozessor ausgefuehrt. ASIP ist die Abkürzung für \glqq Application Specific Instruction Set Processor \grqq. Die Architektur dieses Prozessors ist anwendungsspezifisch optimiert.ASIPs sind meist auf eine gewisse Funktion optimiert. Durch diese Spezialisierung ist es möglich, Einsparungen hinsichtlich Siliziumfläche und Energieverbrauch gegenüber herkömmlichen Prozessoren zu erreichen. MIPS oder auch \glqq Microprocessor Without Interlocked Pipeline Stages\grqq{} ist eine Befehlssatzarchitektur, bei der der Entwickler/Compiler Datenabhängigkeiten der Befehle erkennen und auflösen muss.
Um dem niedrigen Energieverbrauch und Flächenbedarf bei Hörgeräten zu genügen, handelt es sich bei dem Prozessor zusätzlich um eine vereinfachte Implementierung der MOAI-Prozessor-Architektur (siehe Abbildung \ref{fig:KAVUAKA}). Dieser wird auch KAVUAKA genannt und ist ein VLIW-SIMD ASIP Prozessor.\\
	\begin{figure}[H] 
		\centering
		\includesvg[width=0.70\textwidth]{kavuaka2pipeline}
		\caption[KAVUAKA-Prozessor]{KAVUAKA-Prozessor \cite{lukasglitches2017}}
		\label{fig:KAVUAKA}
	\end{figure}
Bei VLIW  handelt es sich um eine Eigenschaft von Mikroprozessor-Architekturen. VLIW steht für \glqq Very Long Instruction Word \grqq und bedeutet soviel wie sehr langes Instruktions-Wort. Das Ziel dieser Struktur ist eine schnelle Abarbeitung des Befehlsatzes, wobei einige Befehle parallel ausgeführt werden können. Um dies zu ermöglichen, sind mehrere Instruktions-Dekoder vonnöten. Eine VLIW-Architekur geht meist mit Pipelining einher.\\
Der Begriff SIMD steht für \glqq Single Instruction Multiple Data \grqq und ist eine Datenverarbeitungseigenschaft von Prozessoren, die meist in der Signalverarbeitung zum Einsatz kommt. Der Vorteil einer solchen Implementierung ist, dass eine Instruktion in mehreren ALUs parallel mit unterschiedlichen Daten ausgeführt werden kann. Somit ist nur eine Instruktion für die Berechnung vieler Operationen nötig, wodurch eine erhebliche Performanz-Steigerung erreichbar ist.\cite[Seite 249]{wust2010mikroprozessortechnik}\\
Außerdem verfügt die Architektur des verwendeten Prozessors über zwei Pipeline-Stufen.
Wie das Wort Pipelining schon besagt, handelt es sich hierbei um eine Befehlsabarbeitung am Fließband. Wurde ein Befehl in der ersten Pipelinestufe abgearbeitet, kann dieser an die nächste Stufe weitergeleitet werden und der nachfolgende Befehl führt die erste Stufe aus. Die Zeit bis ein Befehl alle Piplinestufen passiert hat, nennt man Latenzzeit und diese dauert genau so viele Takte, wie die Pipeline Stufen aufweist (siehe Abbildung \ref{fig:pipeline}). Diese Verzögerung mach sich jedoch nur beim Befüllen der Pipeline bemerkbar. Im Idealfall kann im Anschluss mit jedem Takt ein Befehl beendet werden. Somit ist eine parallele Verarbeitung mehrerer unterschiedlicher Befehle möglich. Dabei ist darauf zu achten, dass die Taktung der Pipelinestufen sich an der Stufe orientiert, welche die längste Ausführungsdauer aufweist. \cite[Seite 204]{wust2010mikroprozessortechnik}
	\begin{figure}[H] 
		\centering
		\includesvg[width=0.70\textwidth]{pipeline}
		\caption{Pipeline-Prinzip}
		\label{fig:pipeline}
	\end{figure}
\newpage
Die beiden Pipelinestufen des verwendeten KUAVAKA-Prozessors sind wie folgt aufgeteilt:
\begin{itemize}
	\item[1.] Instruction Fetch - Decode - Register Access (IF - DE - RA) 
	\item[2.] Execute - Write Back (EX - WB)
\end{itemize}

Des Weiteren wurde der Prozessor so erweitert, dass neue Funktionen sowie Co-Prozessoren eingefügt werden können. Zudem besteht die Möglichkeit des Forwarding bei der Verwendung von temporären Variablen.
Der verwendete KAVUAKA-Prozessor besitzt außerdem zwei Instruktion-Dekoder, auch Issue-Slots genannt, und kann somit zwei Befehle parallel ausführen.
Für die Implementierung wird der Prozessor auf einem ASIC entwickelt und zur Verifikation auf einem FPGA getestet, da dieser eine hohe Flexibilität sowie geringere Entwicklungszeiten bietet und die spätere Hardware eins zu eins abgebildet werden kann.\cite{lukasglitches2017}
Bei ASICs handelt es sich um anwendungsspezifische Schaltungen, die in einer integrierten Schaltung zusammengefügt werden. Hierbei ist die komplette Funktion in Hardware abgebildet und ist somit äußerst klein in der Bauweise, performant und gleichzeitig energie-effizient. ZITAT


\newpage

\subsection{Register File Organisation}
Es besteht eine Vielzahl an Registerorganisationen für Signalprozessoren. Im Allgemeinen lassen sich diese auf vier Grundorganisationen herunterbrechen: zentralisierte, gruppierte, hierarchische und partitionierte Registerorganisation (siehe Abbildung \ref{fig:RegisterOrga}) \cite{paya2010multi}.\newline
\begin{figure}[htbp] 
	\centering
	\includegraphics[width=\textwidth]{fig/Register_orga.png}
	\caption[Register File Organisation]{Register File Organisation –\\ a)zentralisierte Organisation b)gruppierte Organisation c)hierarchische Organisation d)partitionierte Organisation  \cite{paya2010multi}}
	\label{fig:RegisterOrga}
\end{figure}
\newline
Im Folgenden soll nun weiter auf die partitionierte Organisation eingegangen werden. Physikalisch ist der Prozessor mit einem 4kB Register File ausgestattet. Dabei handelt es sich nicht um eine monolithische sondern um eine Multishared Register-File Organisation. Das Register wird in mehrere Teile getrennt, wobei die Anzahl von den Issue-Slots abhängt. Durch diese Aufteilung können Einsparungen in der Logik für die Schreib- und Leseports generiert werden und die Anzahl der Register-File-Ports wird erhöht.\cite{paya2010multi}
\begin{scriptsize}
	\begin{figure}[htbp] 
		\centering
		\includesvg[width=\textwidth]{register_file}
		\caption{Register File Organisation}
		\label{fig:reg_orga}
	\end{figure}
\end{scriptsize}

%	\begin{wrapfigure}{r}{0.5\textwidth}
%		\vspace{-5pt}
%		\begin{center}
%			\includesvg[width=0.48\textwidth]{register_file}
%		\end{center}
%		\caption{Register File Organisation}
%		\label{fig:reg_orga}
%	\end{wrapfigure}
Der verwendete Prozessor besitzt zwei Register-Files mit jeweils 32 64bit-Registern (siehe Abbildung \ref{fig:reg_orga}). Beide Register weisen zwei Lese- und vier Schreibports auf. Dadurch ist es möglich, dass Instruktionen aus dem Issue-Slot 1 in das Register-File 0 schreiben oder lesen und umgekehrt. Außerdem wird durch diese Architektur ein Schreiben und Lesen zweier Instruktionen auf das selbe Register-File möglich gemacht. Des Weiteren können im Falle einer X2-Instruktion beide Ports des Registers verwendet werden. Hierfür ist zu beachten, dass nur ein Issue-Slot eine X2-Instruktion ausführen kann, dabei kann diese jedoch vier Lese-Ports und zwei Schreib-Ports verwenden \cite{paya2010multi}. Die Aufteilung und  Zuordnung der Schreib- und Leseports können aus der Abbildung \ref{fig:reg_orga} und \ref{fig::schreib-port} sowie \ref{lese-port} entnommen werden. Die linken Seiten der Tabellen \ref{fig::schreib-port} und \ref{lese-port} zeigt an, welcher Issue-Slot an welches Regiser-File schreiben will. Die rechte Spalte verdeutlicht, wie die Register-Adressen an die Ports des Registers angelegt werden.

\newpage
\begin{table}[H]
	\centering
	\resizebox{0.9\textwidth}{!}{%
	\begin{minipage}{\textwidth}
	\centering
	\begin{tabular}{cccccccccccccccc}
		\multicolumn{8}{c}{Read-Ports.}                 & \multicolumn{8}{|c}{Register-Ports}                                                               \\
		\multicolumn{4}{c}{0} & \multicolumn{4}{|c}{1} & \multicolumn{4}{|c}{v0} &
		\multicolumn{4}{|c}{v1} \\
		\multicolumn{1}{c}{a} & \multicolumn{1}{c}{b} & \multicolumn{1}{c}{c} & \multicolumn{1}{c}{d} &
		\multicolumn{1}{|c}{a} & \multicolumn{1}{c}{b} & \multicolumn{1}{c}{c} & \multicolumn{1}{c}{d} &
		\multicolumn{1}{|c}{0} & \multicolumn{1}{c}{1} & \multicolumn{1}{c}{2}& \multicolumn{1}{c}{3} &
		\multicolumn{1}{|c}{4} & \multicolumn{1}{c}{5} & \multicolumn{1}{c}{6}& \multicolumn{1}{c}{7} \\
		\hline
		\multicolumn{1}{c}{v0} & \multicolumn{1}{c}{v0} & \multicolumn{1}{c}{-} & \multicolumn{1}{c}{-} &
		\multicolumn{1}{c}{v0} & \multicolumn{1}{c}{v0} & \multicolumn{1}{c}{-} & \multicolumn{1}{c}{-} &
		\multicolumn{1}{|c}{a0} & \multicolumn{1}{c}{b0} & \multicolumn{1}{c}{a1}& \multicolumn{1}{c}{b1} &
		\multicolumn{1}{c}{-} &
		\multicolumn{1}{c}{-} &
		\multicolumn{1}{c}{-}&
		\multicolumn{1}{c}{-} \\
		\multicolumn{1}{c}{v0} & \multicolumn{1}{c}{v0} & \multicolumn{1}{c}{-} & \multicolumn{1}{c}{-} &
		\multicolumn{1}{c}{v0} & \multicolumn{1}{c}{v1} & \multicolumn{1}{c}{-} & \multicolumn{1}{c}{-} &
		\multicolumn{1}{|c}{a0} & \multicolumn{1}{c}{b0} & \multicolumn{1}{c}{a1}& \multicolumn{1}{c}{-} &
		\multicolumn{1}{c}{b1} &
		\multicolumn{1}{c}{-} &
		\multicolumn{1}{c}{-}&
		\multicolumn{1}{c}{-} \\
		\multicolumn{1}{c}{v0} & \multicolumn{1}{c}{v0} & \multicolumn{1}{c}{-} & \multicolumn{1}{c}{-} &
		\multicolumn{1}{c}{v1} & \multicolumn{1}{c}{v0} & \multicolumn{1}{c}{-} & \multicolumn{1}{c}{-} &
		\multicolumn{1}{|c}{a0} & \multicolumn{1}{c}{b0} & \multicolumn{1}{c}{b1}& \multicolumn{1}{c}{-} &
		\multicolumn{1}{c}{a1} &
		\multicolumn{1}{c}{-} &
		\multicolumn{1}{c}{-}&
		\multicolumn{1}{c}{-} \\
		\multicolumn{1}{c}{v0} & \multicolumn{1}{c}{v0} & \multicolumn{1}{c}{-} & \multicolumn{1}{c}{-} &
		\multicolumn{1}{c}{v1} & \multicolumn{1}{c}{v1} & \multicolumn{1}{c}{-} & \multicolumn{1}{c}{-} &
		\multicolumn{1}{|c}{a0} & \multicolumn{1}{c}{b0} & \multicolumn{1}{c}{-}& \multicolumn{1}{c}{-} &
		\multicolumn{1}{c}{a1} &
		\multicolumn{1}{c}{b1} &
		\multicolumn{1}{c}{-}&
		\multicolumn{1}{c}{-} \\
		\multicolumn{1}{c}{v0} & \multicolumn{1}{c}{v1} & \multicolumn{1}{c}{-} & \multicolumn{1}{c}{-} &
		\multicolumn{1}{c}{v0} & \multicolumn{1}{c}{v0} & \multicolumn{1}{c}{-} & \multicolumn{1}{c}{-} &
		\multicolumn{1}{|c}{a0} & \multicolumn{1}{c}{a1} & \multicolumn{1}{c}{b1}& \multicolumn{1}{c}{-} &
		\multicolumn{1}{c}{b0} &
		\multicolumn{1}{c}{-} &
		\multicolumn{1}{c}{-}&
		\multicolumn{1}{c}{-} \\
		\multicolumn{1}{c}{v0} & \multicolumn{1}{c}{v1} & \multicolumn{1}{c}{-} & \multicolumn{1}{c}{-} &
		\multicolumn{1}{c}{v0} & \multicolumn{1}{c}{v1} & \multicolumn{1}{c}{-} & \multicolumn{1}{c}{-} &
		\multicolumn{1}{|c}{a0} & \multicolumn{1}{c}{a1} & \multicolumn{1}{c}{-}& \multicolumn{1}{c}{-} &
		\multicolumn{1}{c}{b1} &
		\multicolumn{1}{c}{b0} &
		\multicolumn{1}{c}{-}&
		\multicolumn{1}{c}{-} \\
		\multicolumn{1}{c}{v0} & \multicolumn{1}{c}{v1} & \multicolumn{1}{c}{-} & \multicolumn{1}{c}{-} &
		\multicolumn{1}{c}{v1} & \multicolumn{1}{c}{v0} & \multicolumn{1}{c}{-} & \multicolumn{1}{c}{-} &
		\multicolumn{1}{|c}{a0} & \multicolumn{1}{c}{b1} & \multicolumn{1}{c}{-}& \multicolumn{1}{c}{-} &
		\multicolumn{1}{c}{a1} &
		\multicolumn{1}{c}{b0} &
		\multicolumn{1}{c}{-}&
		\multicolumn{1}{c}{-} \\
		\multicolumn{1}{c}{v0} & \multicolumn{1}{c}{v1} & \multicolumn{1}{c}{-} & \multicolumn{1}{c}{-} &
		\multicolumn{1}{c}{v1} & \multicolumn{1}{c}{v1} & \multicolumn{1}{c}{-} & \multicolumn{1}{c}{-} &
		\multicolumn{1}{|c}{a0} & \multicolumn{1}{c}{-} & \multicolumn{1}{c}{-}& \multicolumn{1}{c}{-} &
		\multicolumn{1}{c}{-b1} &
		\multicolumn{1}{c}{a1} &
		\multicolumn{1}{c}{b0}&
		\multicolumn{1}{c}{-} \\
		\multicolumn{1}{c}{v1} & \multicolumn{1}{c}{v0} & \multicolumn{1}{c}{-} & \multicolumn{1}{c}{-} &
		\multicolumn{1}{c}{v0} & \multicolumn{1}{c}{v0} & \multicolumn{1}{c}{-} & \multicolumn{1}{c}{-} &
		\multicolumn{1}{|c}{b0} & \multicolumn{1}{c}{a1} & \multicolumn{1}{c}{b1}& \multicolumn{1}{c}{-} &
		\multicolumn{1}{c}{a0} &
		\multicolumn{1}{c}{-} &
		\multicolumn{1}{c}{-}&
		\multicolumn{1}{c}{-} \\
		\multicolumn{1}{c}{v1} & \multicolumn{1}{c}{v0} & \multicolumn{1}{c}{-} & \multicolumn{1}{c}{-} &
		\multicolumn{1}{c}{v0} & \multicolumn{1}{c}{v1} & \multicolumn{1}{c}{-} & \multicolumn{1}{c}{-} &
		\multicolumn{1}{|c}{b0} & \multicolumn{1}{c}{a1} & \multicolumn{1}{c}{-}& \multicolumn{1}{c}{-} &
		\multicolumn{1}{c}{b1} &
		\multicolumn{1}{c}{a0} &
		\multicolumn{1}{c}{-}&
		\multicolumn{1}{c}{-} \\
		\multicolumn{1}{c}{v1} & \multicolumn{1}{c}{v0} & \multicolumn{1}{c}{-} & \multicolumn{1}{c}{-} &
		\multicolumn{1}{c}{v1} & \multicolumn{1}{c}{v0} & \multicolumn{1}{c}{-} & \multicolumn{1}{c}{-} &
		\multicolumn{1}{|c}{b0} & \multicolumn{1}{c}{b1} & \multicolumn{1}{c}{-}& \multicolumn{1}{c}{-} &
		\multicolumn{1}{c}{a1} &
		\multicolumn{1}{c}{a0} &
		\multicolumn{1}{c}{-}&
		\multicolumn{1}{c}{-} \\
		\multicolumn{1}{c}{v1} & \multicolumn{1}{c}{v0} & \multicolumn{1}{c}{-} & \multicolumn{1}{c}{-} &
		\multicolumn{1}{c}{v1} & \multicolumn{1}{c}{v1} & \multicolumn{1}{c}{-} & \multicolumn{1}{c}{-} &
		\multicolumn{1}{|c}{b0} & \multicolumn{1}{c}{-} & \multicolumn{1}{c}{-}& \multicolumn{1}{c}{-} &
		\multicolumn{1}{c}{b1} &
		\multicolumn{1}{c}{a1} &
		\multicolumn{1}{c}{a0}&
		\multicolumn{1}{c}{-} \\
		\multicolumn{1}{c}{v1} & \multicolumn{1}{c}{v1} & \multicolumn{1}{c}{-} & \multicolumn{1}{c}{-} &
		\multicolumn{1}{c}{v0} & \multicolumn{1}{c}{v0} & \multicolumn{1}{c}{-} & \multicolumn{1}{c}{-} &
		\multicolumn{1}{|c}{a1} & \multicolumn{1}{c}{b1} & \multicolumn{1}{c}{-}& \multicolumn{1}{c}{-} &
		\multicolumn{1}{c}{b0} &
		\multicolumn{1}{c}{a0} &
		\multicolumn{1}{c}{-}&
		\multicolumn{1}{c}{-} \\
		\multicolumn{1}{c}{v1} & \multicolumn{1}{c}{v1} & \multicolumn{1}{c}{-} & \multicolumn{1}{c}{-} &
		\multicolumn{1}{c}{v0} & \multicolumn{1}{c}{v1} & \multicolumn{1}{c}{-} & \multicolumn{1}{c}{-} &
		\multicolumn{1}{|c}{a1} & \multicolumn{1}{c}{-} & \multicolumn{1}{c}{-}& \multicolumn{1}{c}{-} &
		\multicolumn{1}{c}{b1} &
		\multicolumn{1}{c}{b0} &
		\multicolumn{1}{c}{a0}&
		\multicolumn{1}{c}{-} \\
		\multicolumn{1}{c}{v1} & \multicolumn{1}{c}{v1} & \multicolumn{1}{c}{-} & \multicolumn{1}{c}{-} &
		\multicolumn{1}{c}{v1} & \multicolumn{1}{c}{v0} & \multicolumn{1}{c}{-} & \multicolumn{1}{c}{-} &
		\multicolumn{1}{|c}{b1} & \multicolumn{1}{c}{-} & \multicolumn{1}{c}{-}& \multicolumn{1}{c}{-} &
		\multicolumn{1}{c}{-a1} &
		\multicolumn{1}{c}{b0} &
		\multicolumn{1}{c}{a0}&
		\multicolumn{1}{c}{-} \\
		\multicolumn{1}{c}{v1} & \multicolumn{1}{c}{v1} & \multicolumn{1}{c}{-} & \multicolumn{1}{c}{-} &
		\multicolumn{1}{c}{v1} & \multicolumn{1}{c}{v1} & \multicolumn{1}{c}{-} & \multicolumn{1}{c}{-} &
		\multicolumn{1}{|c}{-} & \multicolumn{1}{c}{-} & \multicolumn{1}{c}{-}& \multicolumn{1}{c}{-} &
		\multicolumn{1}{c}{b1} &
		\multicolumn{1}{c}{a1} &
		\multicolumn{1}{c}{b0}&
		\multicolumn{1}{c}{a0} \\		
		\multicolumn{8}{c}{X2-Mode} &   \multicolumn{8}{|c}{X2-Mode} \\
		\hline
		\multicolumn{1}{c}{v0} & \multicolumn{1}{c}{v0} & \multicolumn{1}{c}{v0} & \multicolumn{1}{c}{v0} &
		\multicolumn{1}{c}{-} & \multicolumn{1}{c}{-} & \multicolumn{1}{c}{-} & \multicolumn{1}{c}{-} &
		\multicolumn{1}{|c}{a0} & \multicolumn{1}{c}{b0} & \multicolumn{1}{c}{c0}& \multicolumn{1}{c}{d0} &
		\multicolumn{1}{c}{-} &
		\multicolumn{1}{c}{-} &
		\multicolumn{1}{c}{-}&
		\multicolumn{1}{c}{-} \\
		\multicolumn{1}{c}{v0} & \multicolumn{1}{c}{v0} & \multicolumn{1}{c}{v0} & \multicolumn{1}{c}{v1} &
		\multicolumn{1}{c}{-} & \multicolumn{1}{c}{-} & \multicolumn{1}{c}{-} & \multicolumn{1}{c}{-} &
		\multicolumn{1}{|c}{a0} & \multicolumn{1}{c}{b0} & \multicolumn{1}{c}{c0}& \multicolumn{1}{c}{-} &
		\multicolumn{1}{c}{d0} &
		\multicolumn{1}{c}{-} &
		\multicolumn{1}{c}{-}&
		\multicolumn{1}{c}{-} \\
		\multicolumn{1}{c}{v0} & \multicolumn{1}{c}{v0} & \multicolumn{1}{c}{v1} & \multicolumn{1}{c}{v0} &
		\multicolumn{1}{c}{-} & \multicolumn{1}{c}{-} & \multicolumn{1}{c}{-} & \multicolumn{1}{c}{-} &
		\multicolumn{1}{|c}{a0} & \multicolumn{1}{c}{b0} & \multicolumn{1}{c}{d0}& \multicolumn{1}{c}{-} &
		\multicolumn{1}{c}{c0} &
		\multicolumn{1}{c}{-} &
		\multicolumn{1}{c}{-}&
		\multicolumn{1}{c}{-} \\
		\multicolumn{1}{c}{v0} & \multicolumn{1}{c}{v0} & \multicolumn{1}{c}{v1} & \multicolumn{1}{c}{v1} &
		\multicolumn{1}{c}{-} & \multicolumn{1}{c}{-} & \multicolumn{1}{c}{-} & \multicolumn{1}{c}{-} &
		\multicolumn{1}{|c}{a0} & \multicolumn{1}{c}{b0} & \multicolumn{1}{c}{-}& \multicolumn{1}{c}{-} &
		\multicolumn{1}{c}{d0} &
		\multicolumn{1}{c}{c0} &
		\multicolumn{1}{c}{-}&
		\multicolumn{1}{c}{-} \\
		\multicolumn{1}{c}{v0} & \multicolumn{1}{c}{v1} & \multicolumn{1}{c}{v0} & \multicolumn{1}{c}{v0} &
		\multicolumn{1}{c}{-} & \multicolumn{1}{c}{-} & \multicolumn{1}{c}{-} & \multicolumn{1}{c}{-} &
		\multicolumn{1}{|c}{a0} & \multicolumn{1}{c}{c0} & \multicolumn{1}{c}{d0}& \multicolumn{1}{c}{-} &
		\multicolumn{1}{c}{b0} &
		\multicolumn{1}{c}{-} &
		\multicolumn{1}{c}{-}&
		\multicolumn{1}{c}{-} \\
		\multicolumn{1}{c}{v0} & \multicolumn{1}{c}{v1} & \multicolumn{1}{c}{v0} & \multicolumn{1}{c}{v1} &
		\multicolumn{1}{c}{-} & \multicolumn{1}{c}{-} & \multicolumn{1}{c}{-} & \multicolumn{1}{c}{-} &
		\multicolumn{1}{|c}{a0} & \multicolumn{1}{c}{c0} & \multicolumn{1}{c}{-}& \multicolumn{1}{c}{-} &
		\multicolumn{1}{c}{d0} &
		\multicolumn{1}{c}{b0} &
		\multicolumn{1}{c}{-}&
		\multicolumn{1}{c}{-} \\
		\multicolumn{1}{c}{v0} & \multicolumn{1}{c}{v1} & \multicolumn{1}{c}{v1} & \multicolumn{1}{c}{v0} &
		\multicolumn{1}{c}{-} & \multicolumn{1}{c}{-} & \multicolumn{1}{c}{-} & \multicolumn{1}{c}{-} &
		\multicolumn{1}{|c}{a0} & \multicolumn{1}{c}{d0} & \multicolumn{1}{c}{-}& \multicolumn{1}{c}{-} &
		\multicolumn{1}{c}{c0} &
		\multicolumn{1}{c}{b0} &
		\multicolumn{1}{c}{-}&
		\multicolumn{1}{c}{-} \\
		\multicolumn{1}{c}{v0} & \multicolumn{1}{c}{v1} & \multicolumn{1}{c}{v1} & \multicolumn{1}{c}{v1} &
		\multicolumn{1}{c}{-} & \multicolumn{1}{c}{-} & \multicolumn{1}{c}{-} & \multicolumn{1}{c}{-} &
		\multicolumn{1}{|c}{a0} & \multicolumn{1}{c}{-} & \multicolumn{1}{c}{-}& \multicolumn{1}{c}{-} &
		\multicolumn{1}{c}{d0} &
		\multicolumn{1}{c}{c0} &
		\multicolumn{1}{c}{b0}&
		\multicolumn{1}{c}{-} \\
		\multicolumn{1}{c}{v1} & \multicolumn{1}{c}{v0} & \multicolumn{1}{c}{v0} & \multicolumn{1}{c}{v0} &
		\multicolumn{1}{c}{-} & \multicolumn{1}{c}{-} & \multicolumn{1}{c}{-} & \multicolumn{1}{c}{-} &
		\multicolumn{1}{|c}{b0} & \multicolumn{1}{c}{c0} & \multicolumn{1}{c}{d0}& \multicolumn{1}{c}{-} &
		\multicolumn{1}{c}{a0} &
		\multicolumn{1}{c}{-} &
		\multicolumn{1}{c}{-}&
		\multicolumn{1}{c}{-} \\
		\multicolumn{1}{c}{v1} & \multicolumn{1}{c}{v0} & \multicolumn{1}{c}{v0} & \multicolumn{1}{c}{v1} &
		\multicolumn{1}{c}{-} & \multicolumn{1}{c}{-} & \multicolumn{1}{c}{-} & \multicolumn{1}{c}{-} &
		\multicolumn{1}{|c}{b0} & \multicolumn{1}{c}{c0} & \multicolumn{1}{c}{-}& \multicolumn{1}{c}{-} &
		\multicolumn{1}{c}{d0} &
		\multicolumn{1}{c}{a0} &
		\multicolumn{1}{c}{-}&
		\multicolumn{1}{c}{-} \\
		\multicolumn{1}{c}{v1} & \multicolumn{1}{c}{v0} & \multicolumn{1}{c}{v1} & \multicolumn{1}{c}{v0} &
		\multicolumn{1}{c}{-} & \multicolumn{1}{c}{-} & \multicolumn{1}{c}{-} & \multicolumn{1}{c}{-} &
		\multicolumn{1}{|c}{b0} & \multicolumn{1}{c}{d0} & \multicolumn{1}{c}{-}& \multicolumn{1}{c}{-} &
		\multicolumn{1}{c}{c0} &
		\multicolumn{1}{c}{a0} &
		\multicolumn{1}{c}{-}&
		\multicolumn{1}{c}{-} \\
		\multicolumn{1}{c}{v1} & \multicolumn{1}{c}{v0} & \multicolumn{1}{c}{v1} & \multicolumn{1}{c}{v1} &
		\multicolumn{1}{c}{-} & \multicolumn{1}{c}{-} & \multicolumn{1}{c}{-} & \multicolumn{1}{c}{-} &
		\multicolumn{1}{|c}{b0} & \multicolumn{1}{c}{-} & \multicolumn{1}{c}{-}& \multicolumn{1}{c}{-} &
		\multicolumn{1}{c}{d0} &
		\multicolumn{1}{c}{c0} &
		\multicolumn{1}{c}{a0}&
		\multicolumn{1}{c}{-} \\
		\multicolumn{1}{c}{v1} & \multicolumn{1}{c}{v1} & \multicolumn{1}{c}{v0} & \multicolumn{1}{c}{v0} &
		\multicolumn{1}{c}{-} & \multicolumn{1}{c}{-} & \multicolumn{1}{c}{-} & \multicolumn{1}{c}{-} &
		\multicolumn{1}{|c}{c0} & \multicolumn{1}{c}{d0} & \multicolumn{1}{c}{-}& \multicolumn{1}{c}{-} &
		\multicolumn{1}{c}{b0} &
		\multicolumn{1}{c}{a0} &
		\multicolumn{1}{c}{-}&
		\multicolumn{1}{c}{-} \\
		\multicolumn{1}{c}{v1} & \multicolumn{1}{c}{v1} & \multicolumn{1}{c}{v0} & \multicolumn{1}{c}{v1} &
		\multicolumn{1}{c}{-} & \multicolumn{1}{c}{-} & \multicolumn{1}{c}{-} & \multicolumn{1}{c}{-} &
		\multicolumn{1}{|c}{c0} & \multicolumn{1}{c}{-} & \multicolumn{1}{c}{-}& \multicolumn{1}{c}{-} &
		\multicolumn{1}{c}{d0} &
		\multicolumn{1}{c}{b0} &
		\multicolumn{1}{c}{a0}&
		\multicolumn{1}{c}{-} \\
		\multicolumn{1}{c}{v1} & \multicolumn{1}{c}{v1} & \multicolumn{1}{c}{v1} & \multicolumn{1}{c}{v0} &
		\multicolumn{1}{c}{-} & \multicolumn{1}{c}{-} & \multicolumn{1}{c}{-} & \multicolumn{1}{c}{-} &
		\multicolumn{1}{|c}{d0} & \multicolumn{1}{c}{-} & \multicolumn{1}{c}{-}& \multicolumn{1}{c}{-} &
		\multicolumn{1}{c}{c0} &
		\multicolumn{1}{c}{b0} &
		\multicolumn{1}{c}{a0}&
		\multicolumn{1}{c}{-} \\
		\multicolumn{1}{c}{v1} & \multicolumn{1}{c}{v1} & \multicolumn{1}{c}{v1} & \multicolumn{1}{c}{v1} &
		\multicolumn{1}{c}{-} & \multicolumn{1}{c}{-} & \multicolumn{1}{c}{-} & \multicolumn{1}{c}{-} &
		\multicolumn{1}{|c}{-} & \multicolumn{1}{c}{-} & \multicolumn{1}{c}{-}& \multicolumn{1}{c}{-} &
		\multicolumn{1}{c}{d0} &
		\multicolumn{1}{c}{c0} &
		\multicolumn{1}{c}{b0}&
		\multicolumn{1}{c}{a0} \\			
		\multicolumn{1}{c}{-} & \multicolumn{1}{c}{-} & \multicolumn{1}{c}{-} & \multicolumn{1}{c}{-} &
		\multicolumn{1}{c}{v0} & \multicolumn{1}{c}{v0} & \multicolumn{1}{c}{v0} & \multicolumn{1}{c}{v0} &
		\multicolumn{1}{|c}{a1} & \multicolumn{1}{c}{b1} & \multicolumn{1}{c}{c1}& \multicolumn{1}{c}{d1} &
		\multicolumn{1}{c}{-} &
		\multicolumn{1}{c}{-} &
		\multicolumn{1}{c}{-}&
		\multicolumn{1}{c}{-} \\
		\multicolumn{1}{c}{-} & \multicolumn{1}{c}{-} & \multicolumn{1}{c}{-} & \multicolumn{1}{c}{-} &
		\multicolumn{1}{c}{v0} & \multicolumn{1}{c}{v0} & \multicolumn{1}{c}{v0} & \multicolumn{1}{c}{v1} &
		\multicolumn{1}{|c}{a1} & \multicolumn{1}{c}{b1} & \multicolumn{1}{c}{c1}& \multicolumn{1}{c}{-} &
		\multicolumn{1}{c}{d1} &
		\multicolumn{1}{c}{-} &
		\multicolumn{1}{c}{-}&
		\multicolumn{1}{c}{-} \\
		\multicolumn{1}{c}{-} & \multicolumn{1}{c}{-} & \multicolumn{1}{c}{-} & \multicolumn{1}{c}{-} &
		\multicolumn{1}{c}{v0} & \multicolumn{1}{c}{v0} & \multicolumn{1}{c}{v1} & \multicolumn{1}{c}{v0} &
		\multicolumn{1}{|c}{a1} & \multicolumn{1}{c}{b1} & \multicolumn{1}{c}{d1}& \multicolumn{1}{c}{-} &
		\multicolumn{1}{c}{c1} &
		\multicolumn{1}{c}{-} &
		\multicolumn{1}{c}{-}&
		\multicolumn{1}{c}{-} \\
		\multicolumn{1}{c}{-} & \multicolumn{1}{c}{-} & \multicolumn{1}{c}{-} & \multicolumn{1}{c}{-} &
		\multicolumn{1}{c}{v0} & \multicolumn{1}{c}{v0} & \multicolumn{1}{c}{v1} & \multicolumn{1}{c}{v1} &
		\multicolumn{1}{|c}{a1} & \multicolumn{1}{c}{b1} & \multicolumn{1}{c}{-}& \multicolumn{1}{c}{-} &
		\multicolumn{1}{c}{d1} &
		\multicolumn{1}{c}{c1} &
		\multicolumn{1}{c}{-}&
		\multicolumn{1}{c}{-} \\
		\multicolumn{1}{c}{-} & \multicolumn{1}{c}{-} & \multicolumn{1}{c}{-} & \multicolumn{1}{c}{-} &
		\multicolumn{1}{c}{v0} & \multicolumn{1}{c}{v1} & \multicolumn{1}{c}{v0} & \multicolumn{1}{c}{v0} &
		\multicolumn{1}{|c}{a1} & \multicolumn{1}{c}{c1} & \multicolumn{1}{c}{d1}& \multicolumn{1}{c}{-} &
		\multicolumn{1}{c}{b1} &
		\multicolumn{1}{c}{-} &
		\multicolumn{1}{c}{-}&
		\multicolumn{1}{c}{-} \\
		\multicolumn{1}{c}{-} & \multicolumn{1}{c}{-} & \multicolumn{1}{c}{-} & \multicolumn{1}{c}{-} &
		\multicolumn{1}{c}{v0} & \multicolumn{1}{c}{v1} & \multicolumn{1}{c}{v0} & \multicolumn{1}{c}{v1} &
		\multicolumn{1}{|c}{a1} & \multicolumn{1}{c}{c1} & \multicolumn{1}{c}{-}& \multicolumn{1}{c}{-} &
		\multicolumn{1}{c}{d1} &
		\multicolumn{1}{c}{b1} &
		\multicolumn{1}{c}{-}&
		\multicolumn{1}{c}{-} \\
		\multicolumn{1}{c}{-} & \multicolumn{1}{c}{-} & \multicolumn{1}{c}{-} & \multicolumn{1}{c}{-} &
		\multicolumn{1}{c}{v0} & \multicolumn{1}{c}{v1} & \multicolumn{1}{c}{v1} & \multicolumn{1}{c}{v0} &
		\multicolumn{1}{|c}{a1} & \multicolumn{1}{c}{d1} & \multicolumn{1}{c}{-}& \multicolumn{1}{c}{-} &
		\multicolumn{1}{c}{c1} &
		\multicolumn{1}{c}{b1} &
		\multicolumn{1}{c}{-}&
		\multicolumn{1}{c}{-} \\
		\multicolumn{1}{c}{-} & \multicolumn{1}{c}{-} & \multicolumn{1}{c}{-} & \multicolumn{1}{c}{-} &
		\multicolumn{1}{c}{v0} & \multicolumn{1}{c}{v1} & \multicolumn{1}{c}{v1} & \multicolumn{1}{c}{v1} &
		\multicolumn{1}{|c}{a1} & \multicolumn{1}{c}{-} & \multicolumn{1}{c}{-}& \multicolumn{1}{c}{-} &
		\multicolumn{1}{c}{d1} &
		\multicolumn{1}{c}{c1} &
		\multicolumn{1}{c}{b1}&
		\multicolumn{1}{c}{-} \\
		\multicolumn{1}{c}{-} & \multicolumn{1}{c}{-} & \multicolumn{1}{c}{-} & \multicolumn{1}{c}{-} &
		\multicolumn{1}{c}{v1} & \multicolumn{1}{c}{v0} & \multicolumn{1}{c}{v0} & \multicolumn{1}{c}{v0} &
		\multicolumn{1}{|c}{b1} & \multicolumn{1}{c}{c1} & \multicolumn{1}{c}{d1}& \multicolumn{1}{c}{-} &
		\multicolumn{1}{c}{a1} &
		\multicolumn{1}{c}{-} &
		\multicolumn{1}{c}{-}&
		\multicolumn{1}{c}{-} \\
		\multicolumn{1}{c}{-} & \multicolumn{1}{c}{-} & \multicolumn{1}{c}{-} & \multicolumn{1}{c}{-} &
		\multicolumn{1}{c}{v1} & \multicolumn{1}{c}{v0} & \multicolumn{1}{c}{v0} & \multicolumn{1}{c}{v1} &
		\multicolumn{1}{|c}{b1} & \multicolumn{1}{c}{c1} & \multicolumn{1}{c}{-}& \multicolumn{1}{c}{-} &
		\multicolumn{1}{c}{d1} &
		\multicolumn{1}{c}{a1} &
		\multicolumn{1}{c}{-}&
		\multicolumn{1}{c}{-} \\
		\multicolumn{1}{c}{-} & \multicolumn{1}{c}{-} & \multicolumn{1}{c}{-} & \multicolumn{1}{c}{-} &
		\multicolumn{1}{c}{v1} & \multicolumn{1}{c}{v0} & \multicolumn{1}{c}{v1} & \multicolumn{1}{c}{v0} &
		\multicolumn{1}{|c}{b1} & \multicolumn{1}{c}{d1} & \multicolumn{1}{c}{-}& \multicolumn{1}{c}{-} &
		\multicolumn{1}{c}{c1} &
		\multicolumn{1}{c}{a1} &
		\multicolumn{1}{c}{-}&
		\multicolumn{1}{c}{-} \\
		\multicolumn{1}{c}{-} & \multicolumn{1}{c}{-} & \multicolumn{1}{c}{-} & \multicolumn{1}{c}{-} &
		\multicolumn{1}{c}{v1} & \multicolumn{1}{c}{v0} & \multicolumn{1}{c}{v1} & \multicolumn{1}{c}{v1} &
		\multicolumn{1}{|c}{b1} & \multicolumn{1}{c}{-} & \multicolumn{1}{c}{-}& \multicolumn{1}{c}{-} &
		\multicolumn{1}{c}{d1} &
		\multicolumn{1}{c}{c1} &
		\multicolumn{1}{c}{a1}&
		\multicolumn{1}{c}{-} \\
		\multicolumn{1}{c}{-} & \multicolumn{1}{c}{-} & \multicolumn{1}{c}{-} & \multicolumn{1}{c}{-} &
		\multicolumn{1}{c}{v1} & \multicolumn{1}{c}{v1} & \multicolumn{1}{c}{v0} & \multicolumn{1}{c}{v0} &
		\multicolumn{1}{|c}{c1} & \multicolumn{1}{c}{d1} & \multicolumn{1}{c}{-}& \multicolumn{1}{c}{-} &
		\multicolumn{1}{c}{b1} &
		\multicolumn{1}{c}{a1} &
		\multicolumn{1}{c}{-}&
		\multicolumn{1}{c}{-} \\
		\multicolumn{1}{c}{-} & \multicolumn{1}{c}{-} & \multicolumn{1}{c}{-} & \multicolumn{1}{c}{-} &
		\multicolumn{1}{c}{v1} & \multicolumn{1}{c}{v1} & \multicolumn{1}{c}{v0} & \multicolumn{1}{c}{v1} &
		\multicolumn{1}{|c}{c1} & \multicolumn{1}{c}{-} & \multicolumn{1}{c}{-}& \multicolumn{1}{c}{-} &
		\multicolumn{1}{c}{d1} &
		\multicolumn{1}{c}{b1} &
		\multicolumn{1}{c}{a1}&
		\multicolumn{1}{c}{-} \\
		\multicolumn{1}{c}{-} & \multicolumn{1}{c}{-} & \multicolumn{1}{c}{-} & \multicolumn{1}{c}{-} &
		\multicolumn{1}{c}{v1} & \multicolumn{1}{c}{v1} & \multicolumn{1}{c}{v1} & \multicolumn{1}{c}{v0} &
		\multicolumn{1}{|c}{d1} & \multicolumn{1}{c}{-} & \multicolumn{1}{c}{-}& \multicolumn{1}{c}{-} &
		\multicolumn{1}{c}{c1} &
		\multicolumn{1}{c}{b1} &
		\multicolumn{1}{c}{a1}&
		\multicolumn{1}{c}{-} \\
		\multicolumn{1}{c}{-} & \multicolumn{1}{c}{-} & \multicolumn{1}{c}{-} & \multicolumn{1}{c}{-} &
		\multicolumn{1}{c}{v1} & \multicolumn{1}{c}{v1} & \multicolumn{1}{c}{v1} & \multicolumn{1}{c}{v1} &
		\multicolumn{1}{|c}{-} & \multicolumn{1}{c}{-} & \multicolumn{1}{c}{-}& \multicolumn{1}{c}{-} &
		\multicolumn{1}{c}{d1} &
		\multicolumn{1}{c}{c1} &
		\multicolumn{1}{c}{b1}&
		\multicolumn{1}{c}{a1}		
	\end{tabular}
	\caption{\label{lese-port}Lese-Port}
	\end{minipage}}
	
	
	
\end{table}
\begin{table}
\resizebox{\textwidth}{!}{%
	\begin{minipage}{\textwidth}
		\centering
		\begin{tabular}{cccccccc}
			\multicolumn{4}{c}{Issue Slot}                 & \multicolumn{4}{|l}{Register-Ports}                                                               \\ 
			\multicolumn{2}{c}{0} & \multicolumn{2}{|c}{1} & \multicolumn{2}{|c}{v0} & 
			\multicolumn{2}{|c}{v1} \\
			\multicolumn{1}{c}{x0} & \multicolumn{1}{c}{y0} &
			\multicolumn{1}{|c}{x1} & \multicolumn{1}{c}{y1} &
			\multicolumn{1}{|c}{0} & \multicolumn{1}{c}{1} & \multicolumn{1}{|c}{2} & \multicolumn{1}{c}{3} \\
			\hline
			\multicolumn{1}{c}{v0} & \multicolumn{1}{c}{-} &
			\multicolumn{1}{c}{v0} & \multicolumn{1}{c}{-} &
			\multicolumn{1}{|c}{x0} & \multicolumn{1}{c}{x1} & \multicolumn{1}{c}{-} &  \multicolumn{1}{c}{-} \\
			\multicolumn{1}{c}{v0} & \multicolumn{1}{c}{-} &
			\multicolumn{1}{c}{v1} & \multicolumn{1}{c}{-} &
			\multicolumn{1}{|c}{x0} & \multicolumn{1}{c}{-} & \multicolumn{1}{c}{x1} &  \multicolumn{1}{c}{-} \\
			\multicolumn{1}{c}{v1} & \multicolumn{1}{c}{-} &
			\multicolumn{1}{c}{v0} & \multicolumn{1}{c}{-} &
			\multicolumn{1}{|c}{x1} & \multicolumn{1}{c}{-} & \multicolumn{1}{c}{x0} &  \multicolumn{1}{c}{-}\\
			\multicolumn{1}{c}{v1} & \multicolumn{1}{c}{-} &
			\multicolumn{1}{c}{v1} & \multicolumn{1}{c}{-} &
			\multicolumn{1}{|c}{-} & \multicolumn{1}{c}{-} & \multicolumn{1}{c}{x0} &  \multicolumn{1}{c}{x1} \\
			\multicolumn{4}{c}{X2-Mode} &   \multicolumn{4}{|c}{X2-Mode} \\
			\hline
			\multicolumn{1}{c}{v0} & \multicolumn{1}{c}{v0} &
			\multicolumn{1}{c}{-} & \multicolumn{1}{c}{-} &
			\multicolumn{1}{|c}{x0} & \multicolumn{1}{c}{y0} & \multicolumn{1}{c}{-} & \multicolumn{1}{c}{-} \\ 
			\multicolumn{1}{c}{v0} & \multicolumn{1}{c}{v1} &
			\multicolumn{1}{c}{-} & \multicolumn{1}{c}{-} &
			\multicolumn{1}{|c}{x0} & \multicolumn{1}{c}{-} & \multicolumn{1}{c}{y0} & \multicolumn{1}{c}{-} \\ 
			\multicolumn{1}{c}{v1} & \multicolumn{1}{c}{v0} &
			\multicolumn{1}{c}{-} & \multicolumn{1}{c}{-} &
			\multicolumn{1}{|c}{y0} & \multicolumn{1}{c}{-} & \multicolumn{1}{c}{x0} & \multicolumn{1}{c}{-} \\ 
			\multicolumn{1}{c}{v1} & \multicolumn{1}{c}{v1} &
			\multicolumn{1}{c}{-} & \multicolumn{1}{c}{-} &
			\multicolumn{1}{|c}{-} & \multicolumn{1}{c}{-} & \multicolumn{1}{c}{x0} &  \multicolumn{1}{c}{y0} \\
			\multicolumn{1}{c}{-} & \multicolumn{1}{c}{-} &
			\multicolumn{1}{c}{v0} & \multicolumn{1}{c}{v0} &
			\multicolumn{1}{|c}{x1} & \multicolumn{1}{c}{y1} & \multicolumn{1}{c}{-} & \multicolumn{1}{c}{-} \\ 
			\multicolumn{1}{c}{-} & \multicolumn{1}{c}{-} &
			\multicolumn{1}{c}{v0} & \multicolumn{1}{c}{v1} &
			\multicolumn{1}{|c}{x1} & \multicolumn{1}{c}{-} & \multicolumn{1}{c}{y1} & \multicolumn{1}{c}{-} \\ 
			\multicolumn{1}{c}{-} & \multicolumn{1}{c}{-} &
			\multicolumn{1}{c}{v1} & \multicolumn{1}{c}{v0} &
			\multicolumn{1}{|c}{y1} & \multicolumn{1}{c}{-} & \multicolumn{1}{c}{x1} & \multicolumn{1}{c}{-} \\ 
			\multicolumn{1}{c}{-} & \multicolumn{1}{c}{-} &
			\multicolumn{1}{c}{v1} & \multicolumn{1}{c}{v1} &
			\multicolumn{1}{|c}{-} & \multicolumn{1}{c}{-} & \multicolumn{1}{c}{x1} &  \multicolumn{1}{c}{y1} \\               
		\end{tabular}
		\caption{\label{fig::schreib-port}Schreib-Port}
		%	\newline
		%	\newline
		%	\newline
	\end{minipage}}
\end{table}

\newpage
\section{Related Work}

Es besteht bereits eine Vielzahl an Ansätzen zur Verlustleistungsoptimierung von Prozessoren. Diese sind in zwei Kategorien, Optimierung durch Hardware-Anpassungen\cite{wu2000clock, horowitz1994low, hajj1998architectural} oder Software \cite{gebotys1997low, asanovic2000energy, toburen1998instruction, su1995cache }eingeteilt. Hierbei werden mittles Hardware verschiedene Optimierungsansätze verfolgt, wie das dynamische Anpassen der Spannung \cite{horowitz1994low} oder Clock-Gating \cite{wu2000clock}. Hinter dem Begriff Clock-Gating verbirgt sich eine Abschaltung des Clock-Signals in Teilen des Prozessors die nicht genutzt werden. Dadurch lässt sich die Schaltaktivität in vielen Bereichen des Prozessors deutlich minimieren. Das Resultat spiegelt sich in der Leistungsaufnahme wieder, die dabei deutlich sinkt \cite{donno2003clock}. In \cite{asanovic2000energy} wurden neue Instruktionen eingeführt um den Energieverbrauch im Register-File zu senken. Ein weiterer Ansatz über das Scheduling die Verlustleistung zu senken, wurde von\cite{toburen1998instruction} untersucht. Hierbei ist das Ziel, so viele Instruktionen wie möglich in einem Takt unterzubringen, welches darin resultiert, dass der Prozessor eine bessere Leistungsaufnahme aufweist.
Eine andere Herrangehensweise um die Verlustleistung im Register su senken, wurde von \cite{su1995cache} veröffentlicht. Dabei wird mit Hilfe einer Gray-Codierung die Schaltaktivität der Adressen minimiert. Dieser Ansatz ist jedoch nur hilfreich, wenn Daten hintereinander im Register liegen. Mit dem in dieser Arbeit vorgestellten Verfahren wird auch in diesem Fall die Schaltaktivität minimiert.

Der Vorteil einer Optimierung zur Compilezeit liegt darin, dass keinerlei neue Hardware benötigt wird. Dies macht das es äußerst lukrativ, da kein Abtausch zwischen Chipfläche und Leistung stattfinden muss. Die meisten Literaturen beschränken sich bei der Optimierung auf das Scheduling und die Ausführungszeit der Codes. Die Besonderheit des Ansatzes in dieser Arbeit liegt bei der Optimierung der Verlustleistung durch den Scheduler und das leistungsoptimierte zuweisen von Registern.
%Die Hauptkomponenten des Compilers sind das Scheduling und die Register Allokation.

\section{Compiler}
%Überleitung zu Tradeoff von Flache und Register File allokation.
Damit ein Prozessor eine Programmiersprache ausführen kann, sind einige Schritte notwendig. Diese sind im Compiler zusammengefasst und werden im Folgenden kurz erläutertert.\\
Der Code wird vorerst in so genannte \glqq Micro Instructions\grqq(MI) gestückelt. Hierbei ist eine MI die Anweisung, die der Prozessor in einem Taktzyklus ausführen kann. Durch die Verwendung eines VLIW-Prozessors ist es möglich, mehrere Instruktionen parallel auszuführen. Im Falle des KAUVAKA-Prozessors sind dies wie erwähnt zwei Instruktionen. Aus diesem Grund teilt der Compiler nun die MIs in \glqq Micro Operation \grqq (MO) auf. Diese werden anschließend in so genannte \glqq Straight Line Microcode\grqq{} (SLM) aufgeteilt. Ein SLM ist so definiert, dass es nur eine Einsprungstelle und Austrittsstelle gibt. Außerdem dürfen sich keine Sprünge und Verzweigungen innerhalb einer SLM befinden.  Dadurch entsteht ein Baum aus SLMs, die im Anschluss einzeln optimiert werden können. Dieses Verfahren ist nötig, da eine Optimierung über alle Instruktionen eine zu hohe Komplexität aufweisen würde. \cite{landskov1980local}
\subsection{Scheduling}
\label{sec:scheduling}
Das Scheduling ist zuständig für die Anordnung der MIs bzw. MOs die sich in einem SLM befinden. Hierbei geht der Algorithmus so vor, dass er die Anordnung sucht welche den geringsten kritischen Pfad aufweist. Dafür sind verschiedene Ansatzmöglichkeiten implementiert. Die einfachste Methode ist das List-Scheduling. Bei diesem Verfahren wird bei jedem Einfügen eines MOs überprüft, ob eine valide Register-Allokation möglich ist. Können die Register nicht zugeordnet werden, beginnt der Scheduler mit der nächsten MO. Im darauffolgenden Schritt wird nun nochmals versucht die erste MO einzufügen. Dies wird solange wiederholt, bis alle MOs zugeordnet sind.\cite{landskov1980local}
Diese Art von Algorithmus findet jedoch nicht immer eine optimale Lösung und ist gerade für große Programme nicht geeignet. Aus diesem Grund kann optional ein genetischer Algorithmus eingesetzt werden, der die Länge des kritischen Pfades einer SLM reduziert.


\section{Register Allokation}
\label{sec:register allok}
\subsection{Virtuelle Register}
\label{sub:virtuelleR}
Bei virtuellen Registern handelt es sich um Register, die an beliebiger Stelle im Register-File allokiert werden können. Das bedeutet, der Compiler kann selbst entscheiden wo im Registerfile er diese Variable platziert. 
Die Idee dabei ist es, dem Compiler die Aufgabe zu übergeben ein geeignetes Register auszuwählen. Das Codebeispiel \ref{phyReg} zeigt anhand einer einfachen Addition diese Funktion. Da der Prozessor nicht mit Immediates addieren kann, müssen zwei Hilfsvariablen verwendet werden. In diesem Fall wurden die Register V0R0 und V0R1 gewählt. Mithilfe dieser Register kann der Prozessor nun eine Addition in Zeile drei durchführen. Anschließend wird der Code in den Speicher zurück geschrieben.
\renewcommand{\lstlistingname}{Codebeispiel}
\begin{lstlisting}[frame=single, caption={physikalische Register},captionpos=b,label=phyReg]
MVI V0R0 0x100
MVI V0R1 0x101
ADD V0R0 V0R0 V0R1
STORE 0x100 V0R0
\end{lstlisting}
Um nun den Code etwas flexibler zu gestalten, wird dem Compiler überlassen, welches Register er benutzt. Die selbe Addition ist in Codebeispiel \ref{virtReg} mit virtuellen Registern realisiert. Hierbei wird dem Scheduler durch ein X gekennzeichnet, dass es sich um ein virtuelles Register handelt. Dabei wählt der Algorithmus anschließend ein optimales Register aus, so dass sich der Entwickler um diese Aufgabe nicht bemühen muss. Dies hat den Vorteil, dass somit für den Code geeignete und optimale Register ausgewählt werden können. Dabei wird darauf geachtet, dass beide Register-Files gleich ausgelastet sind und das es möglich bleibt X2- oder MAC-Befehle (siehe Kapitel \ref{subsec:x2Mode} ff.) zu allokieren. Außerdem sind die Register in diesem Fall so allokiert, dass die Verlustleistungsaufnahme minimal ist. Wie die Register für eine optimale Verlustleistung ausgewählt werden müssen, wird in dieser Arbeit evaluiert und aufgezeigt.

\begin{lstlisting}[frame=single,caption={virtuelle Register},captionpos=b,label=virtReg]
MV VxR0 0x100
MV VxR1 0x101
ADD VxR0 VxR0 VxR1
STORE 0x100 VxR0
\end{lstlisting}
\subsection{X2 Betriebsmodus}\label{subsec:x2Mode}
Der X2-Betriebsmodus ermöglicht es identischen Instruktionen mit dem selben Opcode, den Registerzugriff zusammenzuführen, wobei sich die Adresse nur im letzten Bit unterscheidet. Dadurch wurde virtuell die Anzahl der Issue-Slots erhöht, ohne dabei die Anzahl der zu decodierenden Anweisungen zu erhöhen. Hierbei wird einem Befehl die doppelte Anzahl an Schreib- und Lese-Registern übergeben. Somit steigt die Zahl der Parameter von drei auf sechs. Bei der Auswahl von Registern gibt es die Vorgabe, dass die erste Instruktion auf gerade Register-Adressen und die zweite auf ungerade Adressen zugreift. \cite{paya2009instruction}
\subsection{MAC Operationen}\label{subsec:macMode}
Die MAC (Multiply and Accumulate) -Operation führt eine Multiplikation zweier Werte aus und akkumuliert diese im Anschluss. Werden bei einer n-bit Implementierungen zwei Werte multipliziert, so kann das Resultat eine Bitbreite von 2n aufweisen. Wird nun eine Architektur mit 64-bit Registern verwendet, müssen zwei Register zur Abspeicherung des Ergebnisses herangezogen werden. Dabei sollte es sich um zwei aufeinanderfolgende Register handeln, wobei diese aus dem selben oder einem benachbarten Register-File stammen können. Eine weitere Besonderheit der MAC-Operation ist, dass der erste Operand in einer geraden Register-Adresse liegen muss. Ein Beispiel für eine valide MAC-Allocation stellt das  Register-Paar V0R0 und V1R0 dar.ZITAT
  

\subsection{Dummy-Register}\label{subsec:dummy}
Dummy-Register sind Register, die im Register-File implementiert sind, jedoch kann der Lese- und Schreibzugriff zur Laufzeit über ein so genanntes Dummy-Control-Register gesteuert werden. Mit Hilfe dieser Funktion können beispielsweise Hilfsvariablen die nur kurze Zeit existieren, nicht an das Register-File zurück geschrieben werden. Dadurch wird ein Schreibzugriff auf das Register überflüssig und erzeugt einen Energiegewinn. ZITAT

\subsection{Address-Isolation}\label{subsec:add_iso}
Eine weitere Komponente des Prozessors ist die Adress-Isolation. Dabei handelt es sich um eine Funktion, welche die Switching-Aktivität des Register-Files verringert. Hierbei werden die Adressen der Schreib- und Leseports solange in einem Flipflop gespeichert, bis eine neue Adresse angelegt wird.\cite{lukasglitches2017}

%\subsection{Clock-Gating}\label{subsec:clock-gate}
%Das Taktnetz ist bei Prozessoren in der Regel sehr groß, muss eine hohe Last treiben und hat sehr viele Schaltvorgänge, dadurch nimmt dieses meist über 40\% der Gesamtleistung auf. Aus diesem Grund wurde das Clock-Gating entwickelt. Hinter dem Begriff verbirgt sich eine Abschaltung des Clock-Signals in Teilen des Prozessors der nicht genutzt werden. Dadurch lässt sich die Schaltaktivität in vielen Bereichen des Prozessors deutlich minimieren. Das Resultat spiegelt sich in der Leistungsaufnahme wieder, die dabei deutlich sinkt.\cite{donno2003clock} In dieser Arbeit wird bis auf weiteres kein Clock-Gating eingesetzt.

\section{Verlustleistung}
\label{sec:verlustleistung}
Unter Verlustleistung in integrierten Schaltungen versteht man die in den Transistoren umgesetzte Leistung, die in Form von Wärme verloren geht.
Es wird in statische \(P_{stat}\) und dynamische \(P_{dyn}\) Verlustleistung unterschieden. \cite[Seite 4 ff.]{flynn2007low} Hierbei wird die Verlustleistung von CMOS Transistoren untersucht.
\subsection{Dynamische Verlustleistung}\label{subsec:dynVerl}
Jedes Mal wenn eine Kapazität geladen oder entladen wird, entsteht eine dynamische Verlustleistung \(P_C\). Eine weitere dynamische Verlustleistung \(P_{SC}\) tritt bei dem Schaltvorgang von CMOS-Transistoren auf. Diese soll anhand eines CMOS-Inverters nun verdeutlicht werden. Beim Umschalten der Pegel entsteht eine kurze Zeitspanne, in der beide Transistoren eine leitende Verbindung aufweisen. In diesem Fall besteht ein Kurzschlussstrom \(I_{SC}\) zwischen Versorgungsspannung \(V_{DD}\) und Masse \(V_{SS}\). Die dynamische Verlustleistung ist proportional zur Schaltaktivität \(\alpha\) und somit auch zur Taktfrequenz $f$.\cite[Seite 4 ff.]{flynn2007low}
\begin{equation}
P_{dyn} = \alpha  C_L  V_{dd}^{2}  f
\label{eq:dynVerlustleistung}
\end{equation}
\subsection{Statische Verlustleistung}\label{subsec:statVerl}
Sobald die Verlustleistung unabhängig von der Taktrate ist, kann diese als statische \(P_{Stat}\) bezeichnet werden. Dies ist der Fall, wenn aufbaubedingt ein konstanter Strom zwischen \(V_{DD}\) und \(V_{SS}\) besteht. Dieser Strom ist unabhängig von der angelegten Gatespannung. Der dadurch auftretende Strom wird Leakagestrom genannt. Mit immer kleiner werdenden Strukturen wird dieser Strom immer bedeutender.\cite[Seite 8]{flynn2007low}

Die gesamte Verlustleistung ist die Summer der drei erwähnten Verluste.
\begin{equation}
\begin{aligned}
P &= P_{ C }+P_{ SC }+P_{ Stat}\\
P &= P_{dyn}+P_{Stat}
\label{eq:verlustleistung}
\end{aligned}
\end{equation}

\section{Genetische Algorithmen}
\label{sec:genetischer_algo}
Genetische Algorithmen wurden ursprünglich entwickelt, um evolutionäre Prozesse aus der Natur nachzuempfinden. Erstmals wurde diese Art von Algorithmen von John Holland 1975 entwickelt und untersucht.
In der Natur müssen sich Lebewesen ständig an ihren Lebensraum anpassen und mit den Problemen der Natur leben. Um dies zu ermöglichen, haben sich Lebewesen über Jahrtausende an ihre Umgebungen angepasst. Der Aufbau, die Fähigkeiten und das Erscheinungsbild eines Lebewesen ist von Geburt an vorgegeben. Diese Information befindet sich  verschlüsselt in den Chromosomen. Eine Evolution ist hierbei die Weitergabe dieser Informationen. Durch das Decodieren der Chromosomen entsteht eine neue Lebensform.
Natürliche Selektion ist dabei die Anpassungsfähigkeit des Lebewesens an den vorgegebenen Lebensraum. Demzufolge überleben bzw. pflanzen sich nur die Generationen fort, welche sich gut an die Umstände der Umgebung angepasst haben. Die Mechanismen hinter der Evolution sind noch nicht komplett entschlüsselt, jedoch sind einige Verfahren bekannt, welche im Weiteren betrachtet werden.
Durch das Vorbild der Natur sollen mit genetischen Algorithmen schwierige Sachverhalte gelöst werden können. Hierbei stellen die Chromosomen eine Abbildung einer Lösung auf ein Problem dar. Durch eine sogenannte Fitness-Funktion kann ermittelt werden, wie gut sich ein Chromosom an das gegebene Problem angepasst hat. Wie auch in der Natur werden einzelne Chromosomen fortgepflanzt und bilden neue Generationen. Betrachtet man eine gewisse Anzahl an Generationen, so spricht man von einer Population. Zu Beginn des Algorithmus startet man mit zufälligen Chromosomen, bis eine bestimmte Anzahl an Generationen entstanden ist. Mit dieser Population kann nun die Fortpflanzung betrieben werden.
Durch die Fortpflanzung werden die Chromosomen zweier Lebewesen an eine neue Generation weitergegeben. Auch der Prozess der Fortpflanzung ist bis heute nicht genau entschlüsselt, jedoch kann der Prozess mittels verschiedener Verfahren angenähert werden. Darunter fällt das Crossover und die Mutation \cite{davis1991handbook}. Nun wird jedoch erst auf das Prinzip der Fitness eingegangen werden. 

\subsection{Fitness}
Die Fitness einer Generation gibt an, wie gut sich das Chromosom an das gegebene Problem angepasst hat. Diese Bewertung ist sehr wichtig für die richtige Funktion des Algorithmus. Dabei muss die Fitness-Funktion so entwickelt werden, dass die Generationen unterscheidbar sind und eine Einordnung in der Population möglich ist.\cite{davis1991handbook}

\subsection{Crossover}
\label{chap:grundlagen_cossover}
Bei dem Crossover-Prozess handelt es sich um die Fortpflanzung von Generationen. Hierbei ist ausschlaggebend welche Generationen miteinander gepaart werden. Auf den ersten Blick scheint es einleuchtend, immer die Generation mit der besten Fitness zu paaren. Dies ist jedoch nicht immer sinnvoll, da so schnell ein lokales Minimum erreicht wird. Aus diesem Grund gibt es verschiedene Ansätze um geeignete Eltern für die neue Generation zu finden.\cite{davis1991handbook}
Die in der Literatur am weitest verbreiteten Methoden sind die Roulette-Wheele-Selection und die Tournament-Selection.\cite{zhong2005comparison} Bei beiden Verfahren besteht die Möglichkeit, dass Populationsmitglieder mit einer niedrigen Fitness selektiert und somit gepaart werden.
% wird zufällig ein Elternpaar gewählt, wobei die Chance einer jeden Generation ausgewählt zu werden proportional zur Fitness ist. Dieses Verfahren trägt ihren Namen daher, dass es einem Roulette-Rad gleicht, wobei das Rad in Stücke unterteilt ist, welche die Größe proportional zur Fitness haben. Die Auswahl kann nun einem Drehen an einem Rad gleichgesetzt werden. Hierbei ist es wahrscheinlicher, dass die Generationen mit höher Fitness ausgewählt werden. Es ist jedoch auch möglich, dass Populationsmitglieder mit niedriger Fitness gepaart werden.

\subsection{Mutation}
Auch der Prozess der Mutation stammt aus der Natur. Bei der Mutation besteht eine geringe Chance, dass Chromosomen verändert werden und Eigenschaften aufweisen die in keinem der Elternteile aufzufinden sind. Hierzu werden in dem durch Crossover erzeugtem Chromosom einzelne Gene durch Zufall verändert. Die Wahrscheinlichkeit einer zufälligen Veränderung ist dabei wie in der Natur, sehr gering.\cite{davis1991handbook}

