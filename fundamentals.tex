\chapter{Grundlagen}
\label{chap:grundlagen}

Bla bla

\section{Verlustleistung}
\label{sec:verlustleistung}

\subsection{Allgemein}
\subsection{Statische Verlustleistung}
\subsection{Dynamische Verlustleistung}

\section{Genetische Algorithmen}
\label{sec:genetischer_algo}
Genetische Algorithmen wurden ursprünglich entwickelt um evolutionäre Prozesse aus der Natur nachzuempfinden. Erstmals wurde diese Art von Algorithmen von John Holland 1975 entwickelt und untersucht.
In der Natur müssen sich Lebewesen ständig an ihren Lebensraum anpassen und mit den Problemen der Natur leben. Um dies zu ermöglichen haben sich Lebewesen über Jahrtausende an ihre Umgebungen angepasst. Der Aufbau, die Fähigkeiten und das Erscheinungsbild eines Lebewesen ist von Geburt an vorgegeben, diese Information befindet sich in den Chromosomen verschlüsselt. Eine Evolution ist hierbei die Weitergabe dieser Informationen. Durch das decodieren der Chromosomen entsteht eine neue Lebensform.
Natürliche Selektion ist hierbei die Anpassungsfähigkeit des Lebewesens an den vorgegebenen Lebensraum. Demzufolge überleben bzw. pflanzen sich nur die Generationen fort, welche sich gut an die Umstände der Umgebung angepasst haben. Die Mechanismen hinter der Evolution sind noch nicht komplett entschlüsselt, jedoch sind einige Verfahren bekannt welche im weiteren betrachtet werden.
Durch das Vorbild der Natur sollen mit genetischen Algorithmen schwierige Sachverhalte lösen können. Hierbei stellen die Chromosomen eine Abbildung einer Lösung auf ein Problem dar. Durch eine sogenannte Fitness-Funktion, kann ermittelt werden wie gut sich ein Chromosom an das gegebene Problem angepasst hat. Wie auch in der Natur müssen werden einzelne Chromosomen fortgepflanzt und bilden neue Generationen. Betrachtet man eine gewisse Anzahl an Generationen so spricht man von einer Population. Zu Beginn des Algorithmus startet man mit zufälligen Chromosomen, bis eine bestimmte Anzahl Generationen entstanden ist. Mit dieser Population kann nun die Fortpflanzung betrieben werden.
Durch die Fortpflanzung werden die Chromosomen zweier Lebewesen an eine neue Generation weitergegeben, auch dieser Prozess ist bis heute nicht genau entschlüsselt jedoch können einige Merkmale abgebildet werden. Darunter fällt das Crossover und die Mutation.
\subsection{Crossover}
Bei dem Crossover-Prozess, handelt es sich um die Fortpflanzung von Generationen. Hierbei ist ausschlaggebend welche Generationen miteinander gepaart werden. Auf den ersten Blick scheint es einleuchtend immer die Generation mit der besten Fitness zu paaren. Dies ist jedoch nicht immer sinnvoll, da so schnell ein lokales Minimum erreicht wird. Aus diesem Grund gibt es verschiedene Ansätze um geeignete Eltern für die neue Generation zu finden. Die am weitesten verbreitete Methode ist die "Roulette Wheel Selection". Hierbei wird zufällig ein Elternpaar gewählt, wobei die Chance einer jeden Generation ausgewählt zu werden proportional zur Fitness ist. Dieses Verfahren trägt ihren Namen daher, dass es einem Roulette-Rad gleicht, wobei das Rad in Stücke unterteilt ist, welche die Größe proportional zur Fitness haben. Die Auswahl kann nun einem drehen am Rad gleichgesetz werden. Hierbei ist es wahrscheinlicher, dass die Generationen mit höher Fitness ausgewählt werden. Es ist jedoch auch möglich, dass Populationsmitglieder mit niedriger Fitness gepaart werden.
 
\subsection{Mutation}

\subsection{Fitness}

\section{SIMD Prozessor}
\label{sec:VLIW}

\subsection{VLIW}
\subsection{Pipelining}
\subsection{Scheduling}
\subsection{XXX}


\section{Register Allokation}
\label{sec:register allok}
\subsection{virtuelle Register}
\subsection{X2 Modus}
\subsection{MAC Modus}