% !TeX spellcheck = de_DE
\chapter{Grundlagen}
\label{chap:grundlagen}

Bla bla
\subsection{MIPS}
MIPS oder auch \glqq Microprocessor without interlocked pipeline stages\grqq{} ist eine Befehlssatzarchitektur, die auf der RISC-Architektur  (Reduced instruction set computer) aufbaut. Hierbei

\section{SIMD Prozessor}
\label{sec:VLIW}
Der Begriff SIMD steht für Single Instruction Multiple Data und ist eine Datenverarbeitungseigenschaft von Prozessoren die meist in der Signalverarbeitung zum Einsatz kommt. Hierbei ist der Vorteil einer solchen Implementierung, dass mehrere Instruktionen parallel mit den selben Daten ausgeführt werden können und somit eine Performanz-Steigung einhergeht.

\subsection{VLIW}
VLIW ist eine Eigenschaft von Mikroprozessor-Architekturen. VLIW steht hierbei für Very Long Instruction Word und bedeutet soviel wie sehr langes Instruktions-Wort. Das Ziel dieser Struktur ist eine schnelle Abarbeitung des Befehlsatzes, wobei hierbei einige Befehle parallel ausgeführt werden. Um dies zu ermöglichen sind mehrere Instruktions-Dekoder vonnöten. Der verwendete KAVUAKA-Prozessor besitzt zwei solcher Dekoder auch Issue-Slots genannt und kann somit zwei Befehle parallel ausführen. Eine VLIW-Architekur geht meist mit Pipelining einher. 
\subsection{Pipelining}
Wie das Wort Pipelining schon besagt, handelt es sich hierbei um eine Befehlsabarbeitung am Fließband(Pipeline). Wurde ein Befehl in Phase 1 abgearbeitet kann dieser an die nächste Phase weitergeleitet werden und der nachfolgende Befehl führt die Phase 1 aus. Somit ist eine parallele Verarbeitung mehrerer Befehle möglich. Die Phasen des in dieser Arbeit verwendeten VLIW-Prozessors sind: 
\begin{itemize}
	\item[1.] Instruction Fetch - Decode(IF - DE) 
	\item[2.] Register Access (RA)
	\item[3.] Execute - Write Back (EX - WB)
\end{itemize}

\newpage

\section{Aufbau der Architektur}
\label{chap:architecture_overview}
Bei dem verwendeten Prozessor handelt es sich um ein MIPS-Prozessor.
KAVUAKA


\begin{scriptsize}
	\begin{figure}[htbp] 
		\centering
		\includesvg[width=0.70\textwidth]{kavuaka2pipeline}
		\caption{KAVUAKA-Prozessor}
		\label{fig:KAVUAKA}
	\end{figure}
\end{scriptsize}


\subsection{Register File Organisation}
Es besteht eine Vielzahl an Registerorganisationen für Signalprozessoren. Im Allgemeinen lassen sich diese auf vier Grundorganisationen herunterbrechen: zentralisierte, gruppierte, hierarische und partitionierte Registerorganisation (siehe Abbildung XXX).
Im folgenden soll nun weiter auf die partitionierte Organisation eingegangen werden. Physikalisch ist der Prozessor mit einem 4kB Register File ausgestattet. Dabei handelt es sich nicht um eine monolithische sondern um ein Multishared Register-File Organisation. Hierbei wird das Register in mehrere Teile getrennt, die Anzahl hängt hierbei von den Issue-Solts ab. Durch diese Aufteilung können Einsparungen in der Logik für die Schreib- und Leseports generiert werden und die Anzahl der Register-File-Ports wird erhöht, dies stellt bei VLIW Prozessoren oft ein Bottelneck dar.\\
In dieser Arbeit werden zwei Register-Files mit jeweils 32 Registern verwendet. Beide Register weissen zwei Lese und vier Schreibports auf. Dadurch ist es möglich, dass Instruktionen aus dem Issue-Slot 1 in das Register-File 0 schreiben oder lesen können und umgekehrt. Außerdem wird durch diese Architektur ein Schreiben und Lesen zweier Instruktionen auf das selbe Register-File möglich gemacht. Des weiteren können im Falle einer X2-Instruktion beide Ports des Registers verwendet werden. Die Aufteilung und  Zuordnung der Schreib- und Leseports können aus der Abbildung \ref{fig:reg_orga} und \autoref{fig::schreib-port}+\ref{lese-port} entnommen werden.

\begin{figure}[htbp] 
	\centering
	\includesvg[width=1.0\textwidth]{register_orga}
	\caption{Register File Organistaion }
	\label{fig:reg_orga}
\end{figure}


\begin{table}[htbp]
	
	\begin{minipage}{.4\textwidth}
		\flushleft
		\begin{tabular}{cccccc}
			\multicolumn{2}{l}{Schreib-Instruktion}                 & \multicolumn{4}{|l}{Schreib-Ports}                                                               \\ 
			\multicolumn{1}{c}{0} & \multicolumn{1}{c}{1} & \multicolumn{1}{|c}{0} & \multicolumn{1}{c}{1} & \multicolumn{1}{c}{2} & \multicolumn{1}{c}{3} \\ 
			\hline
			\multicolumn{1}{c}{0} & \multicolumn{1}{c}{0} & \multicolumn{1}{|c}{0} & \multicolumn{1}{c}{1} & \multicolumn{1}{c}{} & \multicolumn{1}{c}{} \\ 
			\multicolumn{1}{c}{0} & \multicolumn{1}{c}{1} & \multicolumn{1}{|c}{0} & \multicolumn{1}{c}{} & \multicolumn{1}{c}{1} & \multicolumn{1}{c}{} \\ 
			\multicolumn{1}{c}{1} & \multicolumn{1}{c}{0} & \multicolumn{1}{|c}{1} & \multicolumn{1}{c}{} & \multicolumn{1}{c}{0} & \multicolumn{1}{c}{} \\ 
			\multicolumn{1}{c}{1} & \multicolumn{1}{c}{1} & \multicolumn{1}{|c}{} & \multicolumn{1}{c}{} & \multicolumn{1}{c}{0} &  \multicolumn{1}{c}{1}                   
		\end{tabular}
		\caption{\label{fig::schreib-port}Schreib-Port}
	\end{minipage}
	\hfill
	\begin{minipage}{.4\textwidth}
		\flushleft
		
		\begin{tabular}{cccccccccccccccccc}
			\multicolumn{4}{l}{Lese-Inst.}                 & \multicolumn{8}{|l}{Lese-Ports}                                                               \\ 
			\multicolumn{1}{c}{0} & \multicolumn{1}{c}{1} & \multicolumn{1}{c}{2} & \multicolumn{1}{c}{3} & \multicolumn{1}{|c}{0} & \multicolumn{1}{c}{1} & \multicolumn{1}{c}{2}& \multicolumn{1}{c}{3} &
			\multicolumn{1}{c}{4} & \multicolumn{1}{c}{5} & \multicolumn{1}{c}{6}& \multicolumn{1}{c}{7} \\
			\hline
			\multicolumn{1}{c}{0} & \multicolumn{1}{c}{0} & \multicolumn{1}{c}{0} & \multicolumn{1}{c}{0} & \multicolumn{1}{|c}{0} & \multicolumn{1}{c}{1} & \multicolumn{1}{c}{2}& \multicolumn{1}{c}{3} &
			\multicolumn{1}{c}{} & \multicolumn{1}{c}{} & \multicolumn{1}{c}{}& \multicolumn{1}{c}{} \\
			\multicolumn{1}{c}{0} & \multicolumn{1}{c}{0} & \multicolumn{1}{c}{0} & \multicolumn{1}{c}{1} & \multicolumn{1}{|c}{0} & \multicolumn{1}{c}{1} & \multicolumn{1}{c}{2}& \multicolumn{1}{c}{} &
			\multicolumn{1}{c}{3} & \multicolumn{1}{c}{} & \multicolumn{1}{c}{}& \multicolumn{1}{c}{} \\
			\multicolumn{1}{c}{0} & \multicolumn{1}{c}{0} & \multicolumn{1}{c}{1} & \multicolumn{1}{c}{0} & \multicolumn{1}{|c}{0} & \multicolumn{1}{c}{1} & \multicolumn{1}{c}{3}& \multicolumn{1}{c}{} &
			\multicolumn{1}{c}{2} & \multicolumn{1}{c}{} & \multicolumn{1}{c}{}& \multicolumn{1}{c}{} \\
			\multicolumn{1}{c}{0} & \multicolumn{1}{c}{0} & \multicolumn{1}{c}{1} & \multicolumn{1}{c}{1} & \multicolumn{1}{|c}{0} & \multicolumn{1}{c}{1} & \multicolumn{1}{c}{}& \multicolumn{1}{c}{} &
			\multicolumn{1}{c}{3} & \multicolumn{1}{c}{2} & \multicolumn{1}{c}{}& \multicolumn{1}{c}{} \\
			\multicolumn{1}{c}{0} & \multicolumn{1}{c}{1} & \multicolumn{1}{c}{0} & \multicolumn{1}{c}{0} & \multicolumn{1}{|c}{0} & \multicolumn{1}{c}{2} & \multicolumn{1}{c}{3}& \multicolumn{1}{c}{} &
			\multicolumn{1}{c}{1} & \multicolumn{1}{c}{} & \multicolumn{1}{c}{}& \multicolumn{1}{c}{} \\
			\multicolumn{1}{c}{0} & \multicolumn{1}{c}{1} & \multicolumn{1}{c}{0} & \multicolumn{1}{c}{1} & \multicolumn{1}{|c}{0} & \multicolumn{1}{c}{2} & \multicolumn{1}{c}{}& \multicolumn{1}{c}{} &
			\multicolumn{1}{c}{3} & \multicolumn{1}{c}{1} & \multicolumn{1}{c}{}& \multicolumn{1}{c}{} \\
			\multicolumn{1}{c}{0} & \multicolumn{1}{c}{1} & \multicolumn{1}{c}{1} & \multicolumn{1}{c}{0} & \multicolumn{1}{|c}{0} & \multicolumn{1}{c}{3} & \multicolumn{1}{c}{}& \multicolumn{1}{c}{} &
			\multicolumn{1}{c}{2} & \multicolumn{1}{c}{1} & \multicolumn{1}{c}{}& \multicolumn{1}{c}{} \\
			\multicolumn{1}{c}{0} & \multicolumn{1}{c}{1} & \multicolumn{1}{c}{1} & \multicolumn{1}{c}{1} & \multicolumn{1}{|c}{0} & \multicolumn{1}{c}{} & \multicolumn{1}{c}{}& \multicolumn{1}{c}{} &
			\multicolumn{1}{c}{3} & \multicolumn{1}{c}{2} & \multicolumn{1}{c}{1}& \multicolumn{1}{c}{} \\
			\multicolumn{1}{c}{1} & \multicolumn{1}{c}{0} & \multicolumn{1}{c}{0} & \multicolumn{1}{c}{0} & \multicolumn{1}{|c}{1} & \multicolumn{1}{c}{2} & \multicolumn{1}{c}{3}& \multicolumn{1}{c}{} &
			\multicolumn{1}{c}{0} & \multicolumn{1}{c}{} & \multicolumn{1}{c}{}& \multicolumn{1}{c}{} \\
			\multicolumn{1}{c}{1} & \multicolumn{1}{c}{0} & \multicolumn{1}{c}{0} & \multicolumn{1}{c}{1} & \multicolumn{1}{|c}{1} & \multicolumn{1}{c}{2} & \multicolumn{1}{c}{}& \multicolumn{1}{c}{} &
			\multicolumn{1}{c}{3} & \multicolumn{1}{c}{0} & \multicolumn{1}{c}{}& \multicolumn{1}{c}{} \\
			\multicolumn{1}{c}{1} & \multicolumn{1}{c}{0} & \multicolumn{1}{c}{1} & \multicolumn{1}{c}{0} & \multicolumn{1}{|c}{1} & \multicolumn{1}{c}{3} & \multicolumn{1}{c}{}& \multicolumn{1}{c}{} &
			\multicolumn{1}{c}{2} & \multicolumn{1}{c}{0} & \multicolumn{1}{c}{}& \multicolumn{1}{c}{} \\
			\multicolumn{1}{c}{1} & \multicolumn{1}{c}{0} & \multicolumn{1}{c}{1} & \multicolumn{1}{c}{1} & \multicolumn{1}{|c}{1} & \multicolumn{1}{c}{} & \multicolumn{1}{c}{}& \multicolumn{1}{c}{} &
			\multicolumn{1}{c}{3} & \multicolumn{1}{c}{2} & \multicolumn{1}{c}{0}& \multicolumn{1}{c}{} \\
			\multicolumn{1}{c}{1} & \multicolumn{1}{c}{1} & \multicolumn{1}{c}{0} & \multicolumn{1}{c}{0} & \multicolumn{1}{|c}{2} & \multicolumn{1}{c}{3} & \multicolumn{1}{c}{}& \multicolumn{1}{c}{} &
			\multicolumn{1}{c}{1} & \multicolumn{1}{c}{0} & \multicolumn{1}{c}{}& \multicolumn{1}{c}{} \\
			\multicolumn{1}{c}{1} & \multicolumn{1}{c}{1} & \multicolumn{1}{c}{0} & \multicolumn{1}{c}{1} & \multicolumn{1}{|c}{2} & \multicolumn{1}{c}{} & \multicolumn{1}{c}{}& \multicolumn{1}{c}{} &
			\multicolumn{1}{c}{3} & \multicolumn{1}{c}{1} & \multicolumn{1}{c}{0}& \multicolumn{1}{c}{} \\
			\multicolumn{1}{c}{1} & \multicolumn{1}{c}{1} & \multicolumn{1}{c}{1} & \multicolumn{1}{c}{0} & \multicolumn{1}{|c}{3} & \multicolumn{1}{c}{} & \multicolumn{1}{c}{}& \multicolumn{1}{c}{} &
			\multicolumn{1}{c}{2} & \multicolumn{1}{c}{1} & \multicolumn{1}{c}{0}& \multicolumn{1}{c}{} \\
			\multicolumn{1}{c}{1} & \multicolumn{1}{c}{1} & \multicolumn{1}{c}{1} & \multicolumn{1}{c}{1} & \multicolumn{1}{|c}{} & \multicolumn{1}{c}{} & \multicolumn{1}{c}{}& \multicolumn{1}{c}{} &
			\multicolumn{1}{c}{3} & \multicolumn{1}{c}{2} & \multicolumn{1}{c}{1}& \multicolumn{1}{c}{0} \\
			
		\end{tabular}
		\caption{\label{lese-port}Lese-Port}
	\end{minipage}
\end{table}


\subsection{Compiler}
Damit ein Prozessor eine Programmiersprache (optimal) ausführen kann sind eine Schritte notwendig. Diese sind im Compiler zusammengefasst und werden im folgenden kurz erläutertert.\\
Der Code wird vorerst in so genannte Micro Instructions (MI) gestückeltt.  Hierbei ist eine MI eine Anweisung welche der Prozessor in einem Taktzyklus ausführen kann. Durch die Verwendung eines VLIW-Prozessors ist es möglich mehrere Instruktionen parallel auszuführen. Im Falle des KAUVAKA-Prozessors sind dies zwei Instruktionen. Aus diesem Grund teilt der Compiler nun die MIs in sogenannte Micro Operation (MO) auf. Diese werden anschließend in so genannte \"Straight Line Microcode\" (SLM) aufgeteilt. Ein SLM ist hierbei so definiert, dass es nur eine Einsprungstelle und eine Austrittsstelle gibt. Außerdem dürfen sich keine Sprünge und Verzweigungen in einer SLM befinden.  Dadurch entsteht ein Baum aus SLMs, die im Anschluss einzeln optimiert werden können. Dieses Verfahren ist nötig, da eine Optimierung über alle Instruktionen eine zu hohe Komplexität hätte. 
\subsection{Scheduling}
Das Scheduling ist zuständig für die Anordnung der MIs bzw. MOs die sich in einem SLM befinden. Hierbei geht der Algorithmus so vor, dass er die Anordnung sucht die den geringesten Kritischenpfad aufweist. Hierbei gibt es verschiedene Ansatzmöglichkeiten. Die einfachste Methode ist das List-Scheduling. Hierbei wird bei jedem einfügen eines MOs überprüft, ob die zugehörigen Register allokiert werden können, ist dies nicht der Fall wird mit der nächsten MO begonnen. Im darauffolgenden Schritt wird nun nochmals versucht die MO einzufügen. Dies wird solange wiederholt bis alle MOs zugeordnet sind.
Diese Art von Algorithmus findet jedoch nicht immer eine optimale Lösung und ist gerade für große Programme nicht geeignet. Aus diesem Grund wurde zusätzlich ein genetischer Algorithmus eingesetzt der die Länge der SLM reduziert.
\subsection{Hamming-Distanze}
Die Hamming-Distanz ist nach dem amerikanischen Mathematiker Richard Wesley Hamming benannt und gibt ein Maß für die Unterschiedlichkeit zweier Zeichenketten an. Hierbei ist die Hamming-Distanz die Anzahl der unterschiedlichen Stellen der beiden Codeworte.
\begin{equation}
	00110 \text{ und } 00100 -> \text{Hamming-Distanze}= 1
	\label{eq:hammingdistanze}
\end{equation}

\section{Register Allokation}
\label{sec:register allok}
\subsection{Virtuelle Register}
Bei virtuellen Registern handelt es sich um Register die an beliebiger Stelle im Register-File alokiert werden können, das heißt der Compiler kann selbst entscheiden wo im Registerfile er diese Variable platziert. 
Die Idee hierbei ist es, dem Compiler die Aufgabe zu übergeben ein geeignetes Register zu suchen. Das Codebeispiel \ref{phyReg} zeigt anhand einer einfachen Addition diese Funktion. Da der Prozessor nicht mit Registeradressen addieren kann, müssen zwei Hilfsvariablen verwendet werden. In diesem Fall wurden die Register V0R0 und V0R1 gewählt. Mithilfe dieser Register kann der Prozessor nun eine Addition in Zeile drei durchführen.
\renewcommand{\lstlistingname}{Codebeispiel}
\begin{lstlisting}[frame=single, caption={physikalische Register},captionpos=b,label=phyReg]
MV V0R0 0x100
MV V0R1 0x101
ADD V0R0 V0R0 V0R1
STORE 0x100 V0R0
\end{lstlisting}
Um nun den Code etwas flexibler zu gestalten, wird  nun dem Compiler überlassen welches Register er benutzt. Die selbe Addition ist in \ref{virtReg} mit virtuellen Registern realisiert. Hierbei wird dem Compiler durch ein  x gekennzeichnet, dass es sich um ein virtuelles Register handelt. Dieser wählt anschließend ein optimales Register aus, so dass sich der Programmierer um diese Aufgabe nicht bemühen muss. Dies hat den Vorteile, dass somit für den Code geeignete/optimale Register ausgewählt werden können. Hierbei wird darauf geachtet, dass beide Register-Files gleich ausgelastete sind und dass es möglich bleibt X2- oder MAC-Befehle (siehe Kapitel \ref{subsec:x2Mode} ff.) allokiert werden können. Außerdem sind die Register in diesem Fall so allokiert, dass die Verlustleistungsaufnahme minimal ist. Wie die Register für eine optimale Verlustleistung ausgewählt werden müssen wird in dieser Arbeit evaluiert und aufgezeigt.

\begin{lstlisting}[frame=single,caption={virtuelle Register},captionpos=b,label=virtReg]
MV VxR0 0x100
MV VxR1 0x101
ADD VxR0 VxR0 VxR1
STORE 0x100 VxR0
\end{lstlisting}
\subsection{X2 Betriebsmodus}\label{subsec:x2Mode}
Der X2-Betriebsmodus ermöglicht es mehrere Funktionseinheiten zu nutzen ohne dabei die Anzahl der zu decodierenden Anweisungen zu erhöhen. Hierbei wird einem Befehl die doppelte Anzahl an Schreib- und Lese-Registern übergeben, somit steigt die Zahl der Parameter von drei auf sechs. Beide Instruktionen eines X2-Befehl können auf das selben Schreibe-Register zugreifen. Bei den Lese-Registern gibt es die Vorgabe, dass die geraden Instruktionen auf gerade Register zugreifen und die ungeraden Instruktionen auf ungerade.% Ausserdem muessen die Instruktionen aus immer in einem gespiegelten Register
\subsection{MAC Betriebsmodus}\label{subsec:macMode}

\subsection{Dummy-Register}\label{subsec:dummy}
Dummy-Register sind Register die im Register-File implementiert sind, jedoch kann der Lese- und Schreibzugriff zur Laufzeit über ein so genanntes Dummy-Control-Register gesteuert werden. Mit Hilfe dieser Funktion koennen beispielsweise Hilfsvariblen die nur kurze Zeit existieren nicht an das Register-File zurück geschrieben werden und somit den Energieverbrauch senkt. 

\subsection{Address-Isolation}\label{subsec:add_iso}
Eine weitere Komponente des Prozessors ist das Adress-Isolation. Dabei handelt es sich um eine Funktion, welche die Switching-Aktivität des Register-Files verringert. Hierbei werden die Adressen der Schreib- und Leseports solange in einem Flipflop gespeichert bis eine neue Adresse angelegt werden (siehe Abbildung XXX).[Analyzin the Trade of Lukas GerlachS]

\subsection{Clock-Gateing}\label{subsec:clock-gate}

\section{Verlustleistung}
\label{sec:verlustleistung}
Unter Verlustleistung in integrierten Schaltungen versteht man die in den Transistoren umgesetzte Leistung die in Form von Wärme verloren geht.
Hierbei wird in statische \(P_{stat}\) und dynamische Verlustleistung \(P_{dyn}\) unterschieden. 
\subsection{Dynamische Verlustleistung}\label{subsec:dynVerl}
Jedes mal wenn eine Kapazität geladen oder entladen wird, entsteht eine dynamische Verlustleistung\(P_C\). Die zweiter dynamische Verlustleistung\(P_{SC}\) entsteht beim Schaltevorgang von Transistoren insbesondere von Inverter-Schaltungen. In diesem Fall existiert beim Umschalten eine kurze Zeitspanne in der beide Transistoren eine leitende Verbindung aufweisen. In diesem Fall besteht ein Kurzschlussstrom \(I_{SC}\)zwischen Versorgungsspannung \(V_{DD}\) und Masse \(V_{SS}\). Die dynamische Verlustleistung ist proportional zur Schaltaktivität\(\alpha\) und somit auch zur Taktfrequenz $f$.
\begin{equation}
P_{dyn} = \alpha  C_L  V_{dd}^{2}  f
\label{eq:dynVerlustleistung}
\end{equation}
\subsection{Statische Verlustleistung}\label{subsec:statVerl}
Sobald die Verlustleistung unabhängig von der Taktrate wird, kann diese als statische \(P_{Stat}\) bezeichnet werden. Dies ist der Fall, wenn die Transistoren so konzeptioniert sind, dass ein konstanter Strom zwischen\(V_{DD}\) und\(V_{SS}\)besteht. Dieser Strom ist unabhängig von der angelegten Gatespannung. Der dadurch auftretende Strom wird Leakagestrom genannt. Mit immer kleiner werdenden Strukturen wird dieser Strom deutliche bedeutender. In diesem Fall hängt der Strom von der Gatespannung ab und die statische wird eine dynamische Verlustleistung.


Die gesamte Verlustleistung ist die Summer der drei erwähnten Verluste.
\begin{equation}
\begin{aligned}
P &= P_{ C }+P_{ SC }+P_{ Stat}\\
P &= P_{dyn}+P_{Stat}
\label{eq:verlustleistung}
\end{aligned}
\end{equation}

\section{Genetische Algorithmen}
\label{sec:genetischer_algo}
Genetische Algorithmen wurden ursprünglich entwickelt um evolutionäre Prozesse aus der Natur nachzuempfinden. Erstmals wurde diese Art von Algorithmen von John Holland 1975 entwickelt und untersucht.
In der Natur müssen sich Lebewesen ständig an ihren Lebensraum anpassen und mit den Problemen der Natur leben. Um dies zu ermöglichen haben sich Lebewesen über Jahrtausende an ihre Umgebungen angepasst. Der Aufbau, die Fähigkeiten und das Erscheinungsbild eines Lebewesen ist von Geburt an vorgegeben, diese Information befindet sich in den Chromosomen verschlüsselt. Eine Evolution ist hierbei die Weitergabe dieser Informationen. Durch das decodieren der Chromosomen entsteht eine neue Lebensform.
Natürliche Selektion ist hierbei die Anpassungsfähigkeit des Lebewesens an den vorgegebenen Lebensraum. Demzufolge überleben bzw. pflanzen sich nur die Generationen fort, welche sich gut an die Umstände der Umgebung angepasst haben. Die Mechanismen hinter der Evolution sind noch nicht komplett entschlüsselt, jedoch sind einige Verfahren bekannt welche im weiteren betrachtet werden.
Durch das Vorbild der Natur sollen mit genetischen Algorithmen schwierige Sachverhalte lösen können. Hierbei stellen die Chromosomen eine Abbildung einer Lösung auf ein Problem dar. Durch eine sogenannte Fitness-Funktion, kann ermittelt werden wie gut sich ein Chromosom an das gegebene Problem angepasst hat. Wie auch in der Natur müssen werden einzelne Chromosomen fortgepflanzt und bilden neue Generationen. Betrachtet man eine gewisse Anzahl an Generationen so spricht man von einer Population. Zu Beginn des Algorithmus startet man mit zufälligen Chromosomen, bis eine bestimmte Anzahl Generationen entstanden ist. Mit dieser Population kann nun die Fortpflanzung betrieben werden.
Durch die Fortpflanzung werden die Chromosomen zweier Lebewesen an eine neue Generation weitergegeben, auch dieser Prozess ist bis heute nicht genau entschlüsselt jedoch können einige Merkmale abgebildet werden. Darunter fällt das Crossover und die Mutation.
\subsection{Crossover}
Bei dem Crossover-Prozess, handelt es sich um die Fortpflanzung von Generationen. Hierbei ist ausschlaggebend welche Generationen miteinander gepaart werden. Auf den ersten Blick scheint es einleuchtend immer die Generation mit der besten Fitness zu paaren. Dies ist jedoch nicht immer sinnvoll, da so schnell ein lokales Minimum erreicht wird. Aus diesem Grund gibt es verschiedene Ansätze um geeignete Eltern für die neue Generation zu finden. Die am weitesten verbreitete Methode ist die Roulette Wheel Selection. Hierbei wird zufällig ein Elternpaar gewählt, wobei die Chance einer jeden Generation ausgewählt zu werden proportional zur Fitness ist. Dieses Verfahren trägt ihren Namen daher, dass es einem Roulette-Rad gleicht, wobei das Rad in Stücke unterteilt ist, welche die Größe proportional zur Fitness haben. Die Auswahl kann nun einem Drehen an einem Rad gleichgesetzt werden. Hierbei ist es wahrscheinlicher, dass die Generationen mit höher Fitness ausgewählt werden. Es ist jedoch auch möglich, dass Populationsmitglieder mit niedriger Fitness gepaart werden.

\subsection{Mutation}
Auch der Prozess der Mutation stammt aus der Natur. Hierbei besteht eine geringe Chance, dass Chromosomen verändert werden und Eigenschaften aufweisen die in keinem der Elternteile aufzufinden sind. Hierzu werden in dem durch Crossover erzeugtem Chromosom einzelne Gene durch Zufall verändert. Die Wahrscheinlichkeit einer zufälligen Veränderung ist hierbei wie in der Natur sehr gering.
\subsection{Fitness}
Die Fitness einer Generation gibt an wie gut sich das Chromosom an das gegebene Problem angepasst hat. Diese Bewertung ist sehr wichtig für die richtige Funktion des Algorithmus. Dabei muss die Fitness-Funktion so entwickelt werden, dass die Generationen unterscheidbar sind und eine Einordnung in der Population möglich ist.  
