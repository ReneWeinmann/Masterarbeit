% !TeX spellcheck = de_DE
\chapter{Grundlagen}
\label{chap:grundlagen}

Bla bla

\section{Verlustleistung}
\label{sec:verlustleistung}
Unter Verlustleistung in integrierten Schaltungen versteht man die in den Transistoren umgesetzte Leistung die in Form von Wärme verloren geht.
Hierbei wird in statische \(P_{stat}\) und dynamische Verlustleistung \(P_{dyn}\) unterschieden. 
\subsection{Dynamische Verlustleistung}
Jedes mal wenn eine Kapazität geladen oder entladen werden muss, entsteht eine dynamische Verlustleistung\(P_C\). Die zweiter dynamische Verlustleistung\(P_{SC}\) entsteht beim Schaltevorgang von Transistoren insbesondere von Inverter-Schaltungen. In diesem Fall existiert beim Umschalten eine kurze Zeitspanne in der beide Transistoren durchgeschaltet sind. In diesem Fall besteht ein Kurzschlussstrom \(I_{SC}\)zwischen Versorgungsspannung \(V_{DD}\) und Masse \(V_{SS}\). Die dynamische Verlustleistung ist proportional zur Schaltaktivität\(\alpha\) und somit auch zur Taktfrequenz $f$.
\begin{equation}
P_{dyn} = \alpha  C_L  V_{dd}^{2}  f
\label{eq:dynVerlustleistung}
\end{equation}
\subsection{Statische Verlustleistung}
Sobald die Verlustleistung unabhängig von der Taktrate wird, kann diese als statische \(P_{Stat}\) bezeichnet werden. Dies ist der Fall, wenn die Transistoren so konzeptioniert sind, dass ein konstanter Strom zwischen\(V_{DD}\) und\(V_{SS}\)besteht. Dieser Strom ist untabhängig von der angelegten Gatespannung. Der dadurch auftretende Srom wird Leakagestrom genannt. Mit immer kleiner werdenden Strukturen wird dieser Strom deutliche wichtiger. In diesem Fall hängt der Strom von der Gatespannung ab und die statische wird eine dynamische Verlustleistung.


Die gesamte Verlustleistung ist die Summer der drei erwähnten Verlusten.
\begin{equation}
	\begin{aligned}
		P &= P_{ C }+P_{ SC }+P_{ Stat}\\
		P &= P_{dyn}+P_{Stat}
	\label{eq:verlustleistung}
	\end{aligned}
\end{equation}

\section{Genetische Algorithmen}
\label{sec:genetischer_algo}
Genetische Algorithmen wurden ursprünglich entwickelt um evolutionäre Prozesse aus der Natur nachzuempfinden. Erstmals wurde diese Art von Algorithmen von John Holland 1975 entwickelt und untersucht.
In der Natur müssen sich Lebewesen ständig an ihren Lebensraum anpassen und mit den Problemen der Natur leben. Um dies zu ermöglichen haben sich Lebewesen über Jahrtausende an ihre Umgebungen angepasst. Der Aufbau, die Fähigkeiten und das Erscheinungsbild eines Lebewesen ist von Geburt an vorgegeben, diese Information befindet sich in den Chromosomen verschlüsselt. Eine Evolution ist hierbei die Weitergabe dieser Informationen. Durch das decodieren der Chromosomen entsteht eine neue Lebensform.
Natürliche Selektion ist hierbei die Anpassungsfähigkeit des Lebewesens an den vorgegebenen Lebensraum. Demzufolge überleben bzw. pflanzen sich nur die Generationen fort, welche sich gut an die Umstände der Umgebung angepasst haben. Die Mechanismen hinter der Evolution sind noch nicht komplett entschlüsselt, jedoch sind einige Verfahren bekannt welche im weiteren betrachtet werden.
Durch das Vorbild der Natur sollen mit genetischen Algorithmen schwierige Sachverhalte lösen können. Hierbei stellen die Chromosomen eine Abbildung einer Lösung auf ein Problem dar. Durch eine sogenannte Fitness-Funktion, kann ermittelt werden wie gut sich ein Chromosom an das gegebene Problem angepasst hat. Wie auch in der Natur müssen werden einzelne Chromosomen fortgepflanzt und bilden neue Generationen. Betrachtet man eine gewisse Anzahl an Generationen so spricht man von einer Population. Zu Beginn des Algorithmus startet man mit zufälligen Chromosomen, bis eine bestimmte Anzahl Generationen entstanden ist. Mit dieser Population kann nun die Fortpflanzung betrieben werden.
Durch die Fortpflanzung werden die Chromosomen zweier Lebewesen an eine neue Generation weitergegeben, auch dieser Prozess ist bis heute nicht genau entschlüsselt jedoch können einige Merkmale abgebildet werden. Darunter fällt das Crossover und die Mutation.
\subsection{Crossover}
Bei dem Crossover-Prozess, handelt es sich um die Fortpflanzung von Generationen. Hierbei ist ausschlaggebend welche Generationen miteinander gepaart werden. Auf den ersten Blick scheint es einleuchtend immer die Generation mit der besten Fitness zu paaren. Dies ist jedoch nicht immer sinnvoll, da so schnell ein lokales Minimum erreicht wird. Aus diesem Grund gibt es verschiedene Ansätze um geeignete Eltern für die neue Generation zu finden. Die am weitesten verbreitete Methode ist die Roulette Wheel Selection. Hierbei wird zufällig ein Elternpaar gewählt, wobei die Chance einer jeden Generation ausgewählt zu werden proportional zur Fitness ist. Dieses Verfahren trägt ihren Namen daher, dass es einem Roulette-Rad gleicht, wobei das Rad in Stücke unterteilt ist, welche die Größe proportional zur Fitness haben. Die Auswahl kann nun einem Drehen an einem Rad gleichgesetzt werden. Hierbei ist es wahrscheinlicher, dass die Generationen mit höher Fitness ausgewählt werden. Es ist jedoch auch möglich, dass Populationsmitglieder mit niedriger Fitness gepaart werden.
 
\subsection{Mutation}
Auch der Prozess der Mutation stammt aus der Natur. Hierbei besteht eine geringe Chance, dass Chromosomen verändert werden und Eigenschaften aufweisen die in keinem der Elternteile aufzufinden sind. Hierzu werden in dem durch Crossover erzeugtem Chromosom einzelne Gene durch Zufall verändert. Die Wahrscheinlichkeit einer zufälligen Veränderung ist hierbei wie in der Natur sehr gering.
\subsection{Fitness}
Die Fitness einer Generation gibt an wie gut sich das Chromosom an das gegebene Problem angepasst hat. Diese Bewertung ist sehr wichtig für die richtige Funktion des Algorithmus. Dabei muss die Fitness-Funktion so entwickelt werden, dass die Generationen unterscheidbar sind und eine Einordnung in der Population möglich ist.  
\section{SIMD Prozessor}
\label{sec:VLIW}

\subsection{VLIW}
VLIW ist eine Eigenschaft von Mikroprozessor-Architekuren. VLIW steht hierbei fuer Very Long Instruction Word und bedeutet soviel wie sehr langes Instruktionsword. Das Ziel dieser Struktur ist eine schnelle Abarbeitung des Befehlsatzes, wobei hierbei einige Befehle parallel ausgefuehrt werden. Um dies zu ermoeglichen sind mehrere Instruktions-Dekoder vonnoeten. Der verwendete TUKUTURI-Prozessor besitzt zwei solcher Dekoder auch Isue-Slots genannt und kann somit zwei Befehle parallel ausfuehren. Eine VLIW-Architekur geht meist mit Pipelining einher. 
\subsection{Pipelining}

\subsection{Compiler}
Um von einer Programmiersprache (in diesem Fall Assembler) zu einem von dem Prozessor ausfuehrbaren Programm zu gelangen, ist es noetig die Programmiersprache in Maschinencode zu uebersetzten. Dies ist die Aufgabe des Compilers. Der verwendete Compiler stueckelt den Code auf in so genannte Micro Instructions (MI). Hierbei ist eine MI eine Anweisung welche der Prozessor in einem Taktzyklus ausfuehren kann, dabei kann es sich auch um mehrere Micro Operation (MO) handeln. Im Falle der verwendeten MOAI-Architektur handelt es sich um zwei MOs die parallel ausgefuehrt werden koennen. Diese MOs werden wiederum gruppiert in sogenannte Straight Line Microcode (SLM). Ein SLM ist hierbei so definiert, dass es nur eine Einsprungstelle und eine Austrittsstelle gibt. Ausserdem duerfen sich keine Spruenge und Verzweigungen in einer SLM befinden.
\subsection{Scheduling}
\subsection{Hamming-Distanze}
Die Hamming-Distanz ist nach dem amerikanischen Mathematiker Richard Wesley Hamming benannt und gibt ein Mass fuer die Unterschiedlichkeit zweier Zeichenketten an. Hierbei ist die Hamming-Distanz die Anzahl der unterschiedlichen Stellen der beiden Codewoerter.
\begin{equation}
	00110 \text{ und } 00100 -> \text{Hamming-Distanze}= 1
	\label{eq:hammingdistanze}
\end{equation}

\section{Register Allokation}
\label{sec:register allok}
\subsection{Virtuelle Register}
Bei virtuellen Registern handelt es sich um Register die an beliebiger Stelle allokiert werden können, das heisst der Compiler kann selbst entscheiden wo im Registerfile er diese Variable platziert. Dies hat den Vorteile, dass somit fuer den Code geeignete/optimale Register ausgewaehlt werden koennen. Hierbei kann darauf geachtet werden, dass beide Register-Files gleich ausgelastete sind und dass es moeglich beibt X2- oder MAC-Befehle (werden im naechsten kapitel beschrieben) allokiert werden koennen. Ausserdem koennen die Register in diesem Fall so allokiert werden, dass die Verlustleistungsaufnahme minimiert werden. Soll ein virtuelles Register verwendet werden so muss dieses im Code mit einem X gekennzeichnet sein. Die Abbildung XXX veranschaulicht die Instruktionen und die daraus resultierende Register-Allokation. 
\subsection{X2 Modus}
\subsection{MAC Modus}